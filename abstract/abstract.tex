\thispagestyle{empty}
\vspace*{3cm}
  \begin{center}
    \textbf{Resumen}
  \end{center}

  El objetivo principal de este trabajo es demostrar el teorema de Arnold-Liouville, que da una condición suficiente para saber si un \em sistema mecánico hamiltoniano \em  es \em integrable por cuadraturas\em. Con este propósito, definimos y desarrollamos los conceptos necesarios para el teorema, dando unas nociones elementales sobre \em geometría simpléctica \em  y su aplicación a la Mecánica Clásica.\\
  

  \noindent \textbf{Palabras clave}:\emph{ Geometría simpléctica, sistemas integrables, teorema de Arnold-Liouville, flujos hamiltonianos, ecuaciones de Hamilton, derivada de Lie, campos dependientes del tiempo.}\\


  \begin{center}
    \textbf{Abstract}
  \end{center}

The main goal of this work is to prove the Arnold-Liouville theorem, which gives a sufficient condition for a \em hamiltonian mechanical system \em to be \em integrable by quadratures\em. To that end we define and develop the concepts involved in the theorem, giving some elementary notions of \em symplectic geometry \em and its application to Classical Mechanics.\\

\noindent \textbf{Keywords}:\emph{ Symplectic geometry, integrable systems, Arnold-Liouville theorem, hamiltonian flows, Hamilton's equations, Lie derivative, time-dependent vector fields.}\\
\


\unmarkedfntext{2010 \emph{Mathematics Subject Classification.} Primary  37J05, 37J35, 53D05, 58A05, 70H05}

