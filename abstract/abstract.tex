\begin{abstract}

  El objetivo principal de este trabajo es demostrar el teorema de Arnold-Liouville, que da una condición suficiente para saber si un \em sistema mecánico hamiltoniano \em  es \em integrable por cuadraturas\em. A lo largo del texto definimos y desarrollamos todos los conceptos necesarios para el estudio del teorema, dando unas nociones básicas sobre \em geometría simpléctica \em  y su aplicación a la Mecánica Clásica.
  
\noindent \em Palabras clave: \em Geometría simpléctica, sistemas integrables, Arnold-Liouville, flujos hamiltonianos, ecuaciones de Hamilton, derivada de Lie, campos dependientes del tiempo\\

\noindent {\sc Abstract.} The main goal of this work is to prove Arnold-Liouville theorem, which gives a sufficient condition for knowing whether a \em hamiltonian mechanical system \em is \em integrable by quadratures\em. Along the text we define and develop all necessary concepts for the study of the theorem, giving some basic notions about \em symplectic geometry \em and its application to Classical Mechanics.

\noindent \em Keywords: \em Symplectic geometry, integrable systems, Arnold-Liouville, hamiltonian flows, Hamilton's equations, Lie derivative, time-dependent vector fields\\
\

\end{abstract}

