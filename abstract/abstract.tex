\begin{abstract}

En este trabajo realizamos una introducción al estudio de las \em variedades simplécticas \em y los \em flujos hamiltonianos\em, con su aplicación a la mecánica clásica. Una variedad simpléctica es una variedad diferenciable de dimensión par en la que se define una forma diferencial $\omega$ cerrada y no degenerada. Previamente, desarrollamos algunas herramientas de geometría diferencial como la \em derivada de Lie \em y los  \em campos dependientes de un parámetro\em. También damos algunas nociones básicas sobre \em espacios vectoriales simplécticos\em. Para terminar, aplicamos los conceptos trabajados para probar algunos resultados importantes sobre \em sistemas integrables\em.

\noindent \em Palabras clave: \em \nota{ya veremos}\\

\noindent {\sc Abstract.}
In this work we make an introductory study to \em symplectic manifolds \em and \em hamiltonian flows\em, with its application to classical mechanics. A symplectic manifold is an even dimensional differentiable manifold in which we define a differential form $\omega$ which is closed and non degenerate. We begin by developing some differential geometric tools such as \em Lie derivative \em and \em 1-parameter dependent fields\em. Moreover, we give some basic notions about \em symplectic vector spaces\em. Finally, we apply the concepts presented to prove some important results on \em integrable systems\em.

\noindent \em Keywords: \em \nota{we'll see}\\
\

\end{abstract}

