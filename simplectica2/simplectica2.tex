\chapter{Geometría simpléctica y mecánica hamiltoniana}\label{cap:simplectica}
\section{Espacios vectoriales simplécticos}
En esta sección repasamos algunos conceptos de álgebra lineal necesarios para estudiar geometría simpléctica. Introduciremos la noción de \emph{espacio vectorial simpléctico} y veremos algunas de sus propiedades. Muchos de los resultados aquí expuestos pueden leerse más desarrollados en \cite{algebra}.
\begin{defn}
  \em
Un \emph{espacio vectorial simpléctico} es un par ordenado $(V,\omega)$, donde $V$ es un espacio vectorial sobre $\RR$ y 
	\[
	  \omega: V \times V \rightarrow \RR
	\]
	es una forma bilineal antisimétrica no degenerada.
\end{defn}

\paragraph{\bf Clasificación de formas bilineales antisimétricas}
  \begin{enumerate}
    \item Sean $\omega$ una forma bilineal antisimétrica sobre un espacio vectorial $V$ de dimensión finita. Entonces exsiste una base respecto de la cual la matriz asociada a $\omega$ es
      \[
	\left(
	\begin{array}{ccc}
	  0 & -I_n & 0 \\
	  I_n & 0 & 0 \\
	  0 & 0 & 0
	\end{array}\right),
      \]
      donde $I_n$ es la matriz identidad $n \times n$ y $n\leq \tfrac{1}{2}\dim (V)$.
    \item Si $(V,\omega)$ es un espacio vectorial simpléctico de dimensión finita, entonces $\dim(V)=2n$ para cierto $n \in \NN$. En tal caso, se dice que $\mathscr{B}\subset V$ es una \emph{base simpléctica} de $V$ si la matriz asociada a $\omega$ en $\mathscr{B}$ es
\[
  J_n :=
\left(
	\begin{array}{cc}
	  0 & -I_n  \\
	  I_n & 0 
	\end{array}\right).
      \]
    \item Podemos ver la forma bilineal $\omega$ de matriz asociada $J_n$ como una $2$-forma alternada en $V$. Si $\{u_1,\dots,u_n,v_1,\dots,v_n\}$ es una base simpléctica y $\{\varphi_1,\dots, \varphi_n, \psi_1,\dots, \psi_n \}$ es su base dual, entonces es inmediato comprobar que
  \[
    \omega =  \sum_{i=1}^n \psi_i \wedge \varphi_i.
  \]
  \item Llamaremos \emph{$n$-ésima forma simpléctica estándar} a la forma $\Omega_n:\RR^{2n} \rightarrow \RR^{2n}$ de matriz asociada $J_n$ en la base canónica de $\RR^{2n}$. Llamaremos \emph{espacio simpléctico estándar $2n$-dimensional} a $(\RR^{2n},\Omega_n)$. Deducimos también de lo anterior que todo espacio vectorial simpléctico de dimensión $2n$ es isomorfo a $(\RR^{2n},\Omega_n)$.
\end{enumerate}
\paragraph{\bf Aplicaciones simplécticas}\mbox{}

  Sean $(V,\omega)$ y $(V',\omega')$ espacios vectoriales simplécticos. Decimos que una aplicación lineal $f:V \rightarrow V'$ es \emph{simpléctica} si
  \[
    \omega'(f(v),f(w)) = \omega(v,w)
  \]
  para cualesquiera $v,w \in V$. En otras palabras $f^*\omega'=\omega$.

Ahora, si $w\neq 0$ y $f(v)=0$, entonces $\omega(v,w)=0$ y, como $\omega$ es no degenerada, $v=0$, de modo que $\ker f=\{0\}$. Es decir, toda aplicación simpléctica es inyectiva.

En el caso en que $V$ y $V'$ tienen la misma dimensión, entonces $f$ es un isomorfismo que lleva una base simpléctica de $(V,\omega)$ a una base simpléctica de $(V',\omega')$. 

\paragraph{\bf El grupo simpléctico real}\mbox{}

  Sea $(V,\omega)$ un espacio vectorial simpléctico de dimensión $2n$. El conjunto de las aplicaciones lineales simplécticas de $V$ en $V$ es un subgrupo de $\mathrm{GL}(2n)$. Este grupo se conoce como \emph{$n$-ésimo grupo simpléctico real} y se denota por $\mathrm{Sp}(2n,\RR)$.

  Asignando a cada $f\in \mathrm{Sp}(2n,\RR)$ su matriz asociada $A$ en una base simpléctica fijada, el grupo simpléctico real se puede representar por medio de las matrices $2n\times 2n$ que cumplen
  \begin{equation*}
    J_n=A^tJ_n A.
  \end{equation*}
  Estas matrices se dicen \emph{matrices simplécticas}.
  De aquí deducimos inmediatamente, por un razonamiento análogo al que haríamos para matrices ortogonales, que todas las matrices simplécticas tienen determinante $1$ o $-1$. Sin embargo, podemos probar un resultado aún más fuerte sobre el grupo simpléctico real: su \emph{subgrupo especial} (es decir, el subgrupo de las aplicaciones de determinante $1$) coincide con él mismo.
\begin{prop}
  Toda aplicación lineal simpléctica tiene determinante $1$.
\end{prop}
\begin{proof}
  Si $\Omega_n$ es la forma simpléctica canónica, entonces $\Lambda=\Omega_n \wedge \overset{(n)}{\cdots} \wedge \Omega_n$ es una forma de grado máximo, luego, por el teorema del determinante, $f^*(\Lambda)=\det(f) \Lambda$. Ahora, como $f$ es simpléctica, 
  \begin{equation*}
    f^*(\Lambda) = (\Omega_n \circ f) \wedge \overset{(n)}{\cdots} \wedge (\Omega_n \circ f)=\Omega_n \wedge \overset{(n)}{\cdots} \wedge \Omega_n = \Lambda.
  \end{equation*}
  Por tanto, $\det(f)=1$.
\end{proof}
\section{Variedades simplécticas}\label{sec:variedades}
Una vez repasados los conceptos básicos de la geometría lineal simpléctica, estamos preparados para entender lo que son las \emph{variedades simplécticas} y estudiar las propiedades que exhiben.
\begin{defn}
  \em
  Una \emph{variedad simpléctica} es un par $(M,\omega)$ donde $M$ es una variedad diferenciable de dimensión $2n$ y $\omega \in \Gamma^2(M)$ es una $2$-forma cerrada y no degenerada (para todo $x \in M$, $\omega_x$ es no degenerada). 
\end{defn}

De la propia definición de variedad simpléctica podemos ya deducir unas cuantas restricciones para lo que puede y lo que no puede ser una variedad simpléctica. En primer lugar, claramente todas tienen que tener dimensión par, ya que lo hemos impuesto por definición, aunque venía motivado del hecho de que solo en dimensión par podemos tener matrices antisimétricas no degeneradas. Veamos entonces un primer ejemplo de variedad simpléctica, a parte del ya estudiado espacio simpléctico estándar --al que trivialmente se le puede dotar de la estructura de variedad simpléctica-- y del fibrado tangente de una variedad riemanniana, que hemos estudiado en el capítulo \ref{sec:fisica}.

\begin{ejemplo}
  \em
  Podemos dar un ejemplo de variedad simpléctica si consideramos la esfera $\SF^2$ y su forma de área $\omega$, que es cerrada por ser de grado máximo. Si vemos $\SF^2$ sumergida en $\RR^3$ como la esfera de radio 1 y consideramos el radio vector $\nu$ en cada punto de la esfera con coordenadas $(x,y,z)$, la expresión concreta de esta forma en este punto es
  \begin{equation*}
    \omega=\det(\nu,\bullet,\bullet)=x\dd y \wedge \dd z - y \dd x \wedge \dd z + z \dd x \wedge \dd y.
  \end{equation*}
  En efecto, como en la esfera unidad
  \begin{equation*}
    x\dd x + y \dd y + z \dd z = 0
  \end{equation*}
  entonces
  \begin{align*}
    x\omega &= x^2 \dd y \wedge \dd z +x \dd x (-y \dd z + z \dd y)\\
    & = x^2 \dd y \wedge \dd z + (-y \dd y + z \dd z) \wedge (-y \dd z +z \dd y) \\
    & = (x^2+y^2+z^2)\dd y \wedge \dd z, 
  \end{align*}
  y, por un cálculo análogo
  \begin{equation*}
    y \omega = -(x^2+y^2+z^2) \dd x \wedge \dd z
  \end{equation*}
  y
  \begin{equation*}
    z \omega = (x^2+y^2+z^2) \dd x \wedge \dd y.
  \end{equation*}
  Como el miembro de la derecha de cada una de estas expresiones sólo se anula en el plano tangente a la esfera en los puntos en los que se anula el correspondiente $x$, $y$ o $z$ de la izquierda, tenemos que la forma $\omega$ nunca es nula en la esfera. 

  También se puede ver cómo se expresa esta forma en coordenadas esféricas
  \begin{equation*}
    \begin{cases}
      x=\cos \phi \sin \theta \\
      y= \sin \phi \cos \theta \\
      z=\cos \theta.
    \end{cases}
  \end{equation*}
  Tenemos entonces
  \begin{equation*}
    \begin{cases}
    \dd x = -\sin \phi \sin \theta \dd \phi + \cos \phi \cos \theta \dd \theta \\
    \dd y = \cos \phi \sin \theta \dd \phi + \sin \phi \cos \theta \dd \theta \\
    \dd z= -\sin \theta \dd \theta,
  \end{cases}
  \end{equation*}
  \begin{equation*}
    \begin{cases}
      \dd x \wedge \dd y = -\sin \theta \cos \theta \dd \phi \wedge \dd \theta \\
      \dd x \wedge \dd z = \sin \phi \sin ^2 \theta \dd \phi \wedge \dd \theta \\
      \dd y \wedge \dd z = -\cos \phi \sin ^2 \theta \dd \phi \wedge \dd \theta.
    \end{cases}
  \end{equation*}
  Finalmente
  \begin{equation*}
    \omega = -(\cos^2 \phi \sin ^3 \theta + \sin^2\phi \sin ^3 \theta +\sin \theta \cos^2 \theta) \dd \phi \wedge \dd \theta = -\sin \theta \dd \phi \wedge \dd \theta,
  \end{equation*}
  que es la forma de área típica en coordenadas esféricas.
\end{ejemplo}

Vamos a obtener ahora algunas condiciones que tienen que cumplir las variedades simplécticas que se deducen directamente de la definición.

\begin{prop}
  Toda variedad simpléctica es orientable.
\end{prop}
\begin{proof}
  Como $\omega$ es no degenerada, $\omega^n=\omega \wedge \overset{(n)}{\cdots} \wedge \omega$ es una forma de grado máximo nunca nula, luego la variedad es orientable.
\end{proof}
A esta $\omega^n$ se le suele llamar \emph{volumen de Liouville} de $M$ ya que, aunque aquí no lo desarrollemos, está relacionada con el elemento de volumen en $M$ si definieramos en ésta una estructura riemanniana.

\begin{prop}
  Si $(M,\omega)$ es una variedad simpléctica compacta y sin borde, entonces su segundo grupo de cohomología de de Rham $H^2(M)$ es no trivial: $\omega$ no es exacta.
\end{prop}
\begin{proof}
  Consideramos en $M$ el volumen de Liouville $\omega^n$. Ahora, si $\omega=\dd \alpha$, entonces $\omega^n=\dd (\alpha \wedge \omega^{n-1})$. Por el teorema de Stokes
  \begin{equation*}
    \int_M \omega^n=\int_M \dd (\alpha \wedge \omega^{n-1}) = \int_{\partial M=\varnothing} \alpha \wedge \omega^{n-1}=0.
  \end{equation*}
  Pero $M$ es compacta y $\omega^n$ es una forma diferencial de grado máximo, luego $\int_M \omega^n$ no puede ser igual a $0$. Por tanto, $\omega$ es cerrada y no es exacta, luego $[\omega]\neq 0$ y $H^2(M) \neq \{0\}$.
\end{proof}
Como consecuencia del resultado anterior, las esferas de dimensión superior a $2$ no admiten una estructura simpléctica ya que $H^2(\SF^n)={0}$ si $n\geq 3$.

\paragraph{\bf Teorema de Darboux}\mbox{}

Prosiguiendo el estudio de las variedades simplécticas más allá de la propia definición nos encontramos con un resultado muy fuerte, central en toda la geometría simpléctica: las formas simplécticas son localmente constantes. Esto establece una diferenciación clara con la geometría riemanniana, donde existen invariantes locales, como la curvatura, de modo que pedir a una métrica riemanianna que sea localmente constante es como pedirle a la variedad que sea localmente plana. Sin embargo, en geometría simpléctica la situación es distinta ya que no hay invariantes locales. 

\begin{thm}[Darboux]\label{tdarboux}
 Sean $M$ una variedad diferenciable y $\omega$ una $2$-forma cerrada y no degenerada en $M$. Entonces, para todo $x\in M$ existe una carta $(U,\varphi)$ en $x$ tal que $\varphi^* \omega$ tiene coeficientes constantes.
\end{thm}

\begin{proof}
  Como se trata de una cuestión local, podemos suponer que estamos en $\RR^{2n}$.
  Sea $\omega_1=\omega_x$, que tiene coeficientes constantes, y sea
  \begin{equation*}
    \omega_t=t\omega_1 + (1-t) \omega = \omega + t(\omega_1-\omega),
  \end{equation*}
  para cada $t\in \RR$. Como $\omega_{t,x}=\omega_x$, para todo $t \in \RR$, $\omega_t$ es no degenerada en $x$. Por tanto, si $J$ es un intervalo abierto acotado tal que $[0,1]\in J$, existe un entorno $U \subset M$ de $x$ en el que $\omega_t$ es no degenerada para cada $t\in \bar{J}$.
  Ahora, como $\omega_1-\omega$ es cerrada, por el Lema de Poincaré existirá una 1-forma $\alpha$ en tal que $\omega_1-\omega=\dd \alpha$. Además, podemos asumir $\alpha_x=0$.

  Ahora, por ser $\omega_t$ no degenerada, da un isomorfismo entre campos y 1-formas (de forma análoga a como vimos en la sección \ref{sec:fisica}): para $t \in J$, podemos definir el campo dependiente del tiempo $X_t$ tal que $i_{X_t}\omega_t= - \alpha$ y además $X_{t,x}=0$ por ser $\alpha_x=0$. Ahora, si $\varphi:V\rightarrow U$ es el flujo dependiente del tiempo asociado a $X$, entonces $\varphi(t,0,x)=x$ para todo $t\in J$, luego $J\times \left\{ 0 \right\}\times \left\{ x \right\} \subset V$. Como $V$ es abierto en $J\times J\times M$ y $[0,1]$ es compacto, existe un entorno $U_0$ de $x$  tal que $[0,1]\times \left\{ 0 \right\}\times U_0 \subset V$. 
  
  Por tanto, si $\varphi_t = \varphi(t,0,\bullet)$ para cada $t\in [0,1]$ se sigue, por la proposición \ref{derivadadeptiempo} y por las fórmulas de Cartan,
  \begin{align*}
    \frac{d}{dt}(\varphi_t^*\omega_t) & = \varphi_t^* (L_{X_t}\omega_t) + \varphi_t^* \left( \frac{d}{dt}\omega_t \right) = \varphi_t^* (i_{X_t}\dd \omega_t+\dd i_{X_t}\omega_t)+\varphi_t^*(\omega_1-\omega) \\
    & = \varphi_t^*(0-\dd \alpha + \omega_1 - \omega) = 0.
  \end{align*}
  De modo que $\varphi_1^*\omega_1=\varphi_0^*\omega_0=\omega$. Por tanto, $\varphi_{1,*}$ es simpléctica, luego es un isomorfismo, lo que implica que $\varphi_1$ es un difeomorfismo local. Así, reduciendo el entorno si es necesario, encontramos la carta que estábamos buscando: basta tomar $\varphi=\varphi_1^{-1}$, y $\varphi^*\omega=\omega_1$ tiene coeficientes constantes.
\end{proof}

\begin{corol}
  Sea $(M,\omega)$ una variedad simpléctica. Para cada $x\in M$ existe un entorno $U$ de $x$ y unas coordenadas $(q,p)=(q_1,\dots,q_n,p_1,\dots,p_n)$ en $x$ tales que 
\begin{equation*}
  \omega= \sum_{i=1}^n \dd p_i \wedge \dd q_i.
\end{equation*}
Una carta $(U,(q,p))$ de este tipo se llama \emph{carta de Darboux}.
\end{corol}
\begin{proof}
  Basta tomar la carta $(U,\varphi)$, dada por el teorema de Darboux, para la cual la forma $\varphi^*\omega$ es constante. Tomamos ahora la aplicación lineal $A:\RR^{2n}\rightarrow \RR^{2n}$, que diagonaliza $\varphi^*\omega$ a la forma $\Omega_n$, y las coordenadas $\xx=(q,p)$ para las cuales el siguiente diagrama conmuta
  \begin{center}
    \begin{tikzcd}
      U \arrow{r}{\varphi}\arrow{rd}[anchor=north,rotate=-30]{\xx} & \RR^{2n}\arrow{d}[anchor=west]{A} \\ 
       &\RR^{2n}.
     \end{tikzcd}
   \end{center}
   Entonces en la carta $(U,\xx)$ podemos escribir $\omega=\xx^*\Omega_n=\sum_{i=1}^n \dd p_i \wedge \dd q_i$.
\end{proof}

\paragraph{\bf Simplectomorfismos}\mbox{}

Para terminar la sección, vamos a estudiar lo que podríamos tomar como morfismos en una categoría de las variedades simplécticas: aplicaciones diferenciables que preserven las $2$-formas.
\begin{defn}
  \em
  Sean $(M,\omega)$ y $(M',\omega')$ variedades simplécticas y $f:M \rightarrow M'$ una aplicación diferenciable. Decimos que $f$ es un \emph{simplectomorfismo} si
    $f^* \omega' = \omega$,
  es decir, si para cada $x\in M$, $(d_xf)^*\omega'_{f(x)}=\omega_x$. Equivalentemente, $f$ es un simplectomorfismo si la aplicación lineal $d_xf:T_xM \rightarrow T_{f(x)}M'$ es simpléctica.
\end{defn}

Una propiedad básica de los simplectomorfismos es que preservan las cartas de Darboux.
\begin{prop}
  Sean $(M,\omega)$ y $(M',\omega')$ variedades simplécticas y $f:M \rightarrow M'$ una aplicación diferenciable. Entonces $f$ es un simplectomorfismo si y sólo si, para cada $x \in M$, si $(U,(q,p))$ es una carta de Darboux en $f(x)$, entonces $(f^{-1}(U),(q\circ f, p \circ f))$ es una carta de Darboux en $x$.
\end{prop}
\begin{proof}\leavevmode
Basta darse cuenta de que
\begin{equation*}
  \sum_{i=1}^n \dd (p_i \circ f) \wedge \dd (q_i \circ f) = \sum_{i=1}^n f^* (\dd p_i \wedge \dd q_i) = f^* \left( \sum_{i=1}^n \dd p_i \wedge \dd q_i \right) = f^* \omega'
\end{equation*}
es igual a $\omega$ si y sólo si $f$ es un simplectomorfismo.

\end{proof}

Enunciamos también una consecuencia inmediata de la definición de simplectomorfismo:
\begin{prop}\label{volumen}
  Los simplectomorfismos preservan el volumen de Liouville $\omega^n$.
\end{prop}

\section{Campos simplécticos y campos hamiltonianos}\label{sec:hamilton}
Sea $(M,\omega)$ una variedad simpléctica. Análogamente a lo visto en la sección \ref{sec:fisica} y en la demostración del teorema de Darboux, la aplicación 
  \begin{equation*}
    \begin{array}{rcl}
      T_xM & \longrightarrow & (T_xM)^* \\
      \xi & \longmapsto & \omega(\bullet,\xi),
      \end{array} 
  \end{equation*}
es un isomorfismo lineal entre campos y 1-formas. Resulta que a cada campo $X$ le podemos asignar la forma $-i_X \omega$. Veremos enseguida como esta propiedad, junto con la fórmula de Cartan nos va a permitir obtener resultados muy interesantes.

En primer lugar, observemos que si $f_t$ es una familia uniparamétrica de simplectomorfismos y $X$ es su generador infinitesimal, entonces
\begin{equation*}
  L_{X}\omega=\left.\frac{d}{dt}\right|_{t=0}f_t^*\omega=0.	
\end{equation*}

  En la otra dirección también es cierto, si $X$ es un campo con $L_X\omega=0$, su flujo $\varphi$ es una familia uniparamétrica de simplectomorfismos. 
  En efecto,
  \begin{align*}
    \frac{d}{dt}(\varphi^*_t\omega_{\varphi_t(x)})&=\lim_{h\rightarrow 0}\frac{\varphi^*_{t+h}\omega_{\varphi_{t+h}(x)}-\varphi^*_t\omega_{\varphi_t(x)}}{h}\\ &=\varphi^*_t\left( \lim_{h\rightarrow 0}\frac{\varphi^*_h\omega_{\varphi_h(\varphi_t(x))}-\omega_{\varphi_t(x)}}{h}\right)\\ &=\varphi_t^*((L_X\omega)_{\varphi_t(x)})=\varphi^*_t(0)=0. 
  \end{align*}
  Por tanto, $\varphi^*_t\omega_{\varphi_t(x)}=\omega_x$.

Por otro lado, aplicando la fórmula de Cartan
\begin{equation*}
  L_X\omega=i_X(\dd\omega)+\dd(i_X\omega)=\dd(i_X\omega),
\end{equation*}
ya que $\omega$ es cerrada. Por tanto, si $X$ es un campo tangente a una variedad simpléctica, $L_X\omega=0$ si y sólo si $i_X\omega$ es cerrada. Todo esto nos lleva a dar la siguiente definición:

\begin{defn}
  \em
  Sea $(M,\omega)$ una variedad simpléctica. Un campo $X\in \mathfrak{X}(M)$ se dice \emph{simpléctico}, si $L_X\omega=0$ o, equivalentemente, si $i_X\omega$ es una 1-forma cerrada. El conjunto de los campos simplécticos de $(M,\omega)$ se denota por $\mathfrak{X}_{\omega}(M)$.

  En el caso en que $X$ sea un campo simpléctico y además $i_X\omega$ sea exacta, existe una función $F:M\rightarrow \RR$ con $i_X\omega=-\dd F$. Denotamos entonces $X=X^F$ y decimos que $X^F$ es un \emph{campo hamiltoniano} con hamiltoniano $F$.
\end{defn}

Localmente, si tomamos una carta de Darboux en un punto $x\in M$, podemos escribir $x=(q_1,\dots,q_n,p_1,\dots p_n)$ y una función $F:M\rightarrow \RR$ como $F(q_1,\dots,q_n,p_1,\dots,p_n)$. Tenemos entonces
\begin{equation*}
  \dd F=\sum_{i=1}^n \parcial{F}{q_i}\dd q_i + \parcial{F}{p_i} \dd p_i,
\end{equation*}
de modo que el campo de hamiltoniano $F$ tiene la forma
\begin{equation*}
    X^F = \sum_{i=1}^n X_{q_i}^F \deriv{q_i} + X_{p_i}^F \deriv{p_i}.
\end{equation*}
  Ahora, 
  \begin{equation*}
    \dd F =- i_{X^F}\omega=- \sum_{i,j=1}^n \dd p_i \wedge \dd q_i \left(X_{q_j}^F \deriv{q_j} + X^F_{p_j} \deriv{p_j},\bullet\right)  = \sum_{i=1}^n X_{q_i}^F \dd p_i -X_{p_i}^F \dd q_i.
  \end{equation*}
  Por tanto, las componentes del campo son $X^F_{q_i}=\parcial{F}{p_i}$ y $X^F_{p_i}=-\parcial{F}{q_i}$. Si consideramos ahora las curvas integrales $(q_1(t),\dots,q_n(t),p_1(t),\dots,p_n(t))$ del campo $X^F$, obtenemos ecuaciones de Hamilton con hamiltoniano $F$, tal y como las vimos en la sección \ref{sec:fisica}
  \begin{equation*}
  \begin{cases}
     \dot{q_i}(t) & = \parcial{F}{p_i}(t) \\
     \dot{p_i}(t) & = -\parcial{F}{q_i} (t),
  \end{cases}
\end{equation*}
  para $i=1,\dots,n$.

  Cabe preguntarse ahora cuándo los campos simplécticos y los campos hamiltonianos coincidirán. Esto es fácil de ver. Si consideramos las aplicaciones $F\mapsto X^F$ y $X\mapsto [i_X\omega]$, nótese que existe una sucesión exacta de espacios vectoriales
  \begin{center}
    \begin{tikzcd}
      \mathscr{C}^{\infty}(M)\arrow{r}{X^\bullet} & \mathfrak{X}_{\omega}(M)  \arrow{r}{[i_{\bullet}\omega]} & H^1(M),  
    \end{tikzcd}
  \end{center}
  donde precisamente $\ker([i_{\bullet}\omega])=\mathrm{im}(X^{\bullet})$ es el conjunto de campos hamiltonianos.
  Si $H^1(M)=0$, entonces todo campo simpléctico es hamiltoniano. Como consecuencia, por el lema de Poincaré, todo campo simpléctico es localmente hamiltoniano. Es decir, si $X$ es un campo simpléctico, en todo punto $x\in M$ podemos tomar un entorno $U$ estrellado en el que hay una función $F:U\rightarrow M$ tal que $-i_X\omega=\dd F$ en $U$.

\section{Corchete de Poisson}\label{sec:poisson}
Podemos emplear los campos hamiltonianos para definir una nueva estructura en las variedades simplécticas, el \emph{corchete de Poisson}. En esta sección veremos que toda variedad simpléctica es una \emph{variedad de Poisson}.

\begin{defn}
  \em
  Sea $(M,\omega)$ una variedad simpléctica y $F,G: M \rightarrow \RR$. Se define el \emph{corchete de Poisson de $F$ y $G$} como la función
\begin{equation*}
  \pois{F}{G}(x) = (X^F G)(x),
\end{equation*}
para cada $x \in M$.
\end{defn}
\begin{prop}\leavevmode
  Sean $F,G:M\rightarrow \RR$. Entonces,
  \begin{enumerate}
    \item[$1$.] $\pois{F}{G} = \dd G (X^F)= \omega(X^F, X^G)$ y
    \item[$2$.] $\lie{X^F}{X^G} = X^{\pois{F}{G}}$. \label{proppoisson}
  \end{enumerate}
\end{prop}
\begin{proof}\leavevmode
  \begin{enumerate}
    \item $\pois{F}{G}=X^F(G)=\dd G(X^F) = -i_{X^G}\omega(X^F)=-\omega(X^G,X^F)=\omega(X^F,X^G)$.
    \item Por las fórmulas de Cartan:
      \begin{align*}
	i_{\lie{X^F}{X^G}} \omega & =  L_{X^F}(i_{X^G}\omega) - i_{X^G}(L_{X^F} \omega) \\ 
	& = \dd i_{X^F}i_{X^G}\omega + i_{X^F}\dd i_{X^G}\omega - i_{X^G}\dd i_{X^F}\omega - i_{X^G}i_{X^F}\dd \omega \\
	& = \dd (\omega(X^G,X^F)) - i_{X^F}(\dd(\dd G)) + i_{X^G}(\dd (\dd F)) - i_{X^G}i_{X^F}(0) \\
	& = \dd \pois{G}{F} = -i_{X^{\pois{G}{F}}}\omega=i_{X^{\pois{F}{G}}}\omega.
      \end{align*}
  \end{enumerate}
\end{proof}

Localmente, si tomamos una carta de Darboux $(q,p)$ en un punto $x\in M$, podemos escribir
  \begin{align*}
    \pois{F}{G} & = \dd G (X^F) = \left( \sum_{i=1}^n \parcial{G}{q_i} \dd q_i + \parcial{G}{p_i} \dd p_i \right) \left( \sum_{i=1}^n \parcial{F}{p_i} \deriv{q_i} - \parcial{F}{q_i} \deriv{p_i}\right) \\ 
    & =\sum_{i=1}^n\left( \parcial{F}{p_i} \parcial{G}{q_i} - \parcial{F}{q_i} \parcial{G}{p_i}\right).
  \end{align*}
  Ésta es la forma del corchete de Poisson que se suele ver en mecánica clásica. Observamos también las relaciones
  \begin{equation*}
    \begin{cases}
    \pois{F}{q_i} &= \parcial{F}{p_i} \\
    \pois{F}{p_i} &= -\parcial{F}{q_i}.
  \end{cases}
  \end{equation*}
Éstas son precisamente las componentes del campo $X^F$. Si ahora consideramos las curvas integrales de $F$, podemos ver que las ecuaciones de Hamilton toman una forma más simple
  \begin{equation*}
    \begin{cases}
      \dot{q_i} &= \pois{F}{q_i} \\
      \dot{p_i} &= \pois{F}{p_i}.
  \end{cases}
  \end{equation*}
  De aquí también obtenemos lo que se conoce como las \emph{relaciones de conmutación canónicas}:
  \begin{equation*}
    \pois{p_i}{q_j}=\delta_{ij}.
  \end{equation*}

  Recíprocamente, si $(U,(q,p))$ es una carta en un punto $x\in M$ tal que, para cada $i,j=1,\dots,n$, $\pois{q_i}{q_j}=\pois{p_i}{p_j}=0$ y $\pois{q_i}{p_j}=\delta_{ij}$, entonces es una carta de Darboux. En efecto, como todos los corchetes de Poisson entre las coordenadas son constantes, entonces $\left[ X^{q_i},X^{q_j} \right]=\left[ X^{p_i},X^{p_j} \right]=\left[ X^{q_i},X^{p_j} \right]=0$, de modo que los campos $X_{p_1},\dots,X_{p_n},X_{q_1},\dots,X_{q_n}$ son coordenados. Esto es, reduciendo el entorno si es preciso, existen unas coordenadas $(\xx,\yy)$ tales que $X^{p_i}=\deriv{\xx_i}$ y $X^{q_i}=-\deriv{\yy_i}$. Ahora, $$\omega\left(\deriv{\xx_i},\deriv{\xx_j}\right)=\omega\left( X^{p_i},X^{p_j} \right)=\left\{ p_i,p_j \right\}=0,$$ $$\omega\left( \deriv{\yy_i},\deriv{\yy_j} \right)=-\omega\left( X^{q_i},X^{q_j} \right)=-\left\{ q_i,q_j \right\}=0$$ y $$\omega\left( \deriv{\xx_i},\deriv{\yy_j} \right)=\omega\left( X^{p_i},-X^{q_j} \right)=-\left\{ p_i,q_j \right\}=-\delta_{ij}.$$ 
  De modo que la carta $(U,(\xx,\yy))$ es de Darboux. Ahora, $$\dd q_i=-i_{X^{q_i}}\omega=i_{\deriv{\yy_i}}\omega=\dd \xx_i$$ y $$\dd p_i=-i_{X^{p_i}}\omega=-i_{\deriv{\xx_i}}\omega=\dd \yy_i,$$ luego $\xx_i=q_i$ e $\yy_i=p_i$ (salvo tal vez una constante). Por tanto, la carta $(U,(q,p))$ es de Darboux.

  Veamos ahora las propiedades del corchete de Poisson.
  \begin{prop}
    La aplicación $\pois{ }{ }:\mathscr{C}^{\infty}(M)\times\mathscr{C}^{\infty}(M) \rightarrow \mathscr{C}^{\infty}(M)$ que a cada par de funciones le asigna su corchete de Poisson cumple las siguientes propiedades:
\begin{enumerate}
  \item[$1.$] Es bilineal,
  \item[$2.$] es antisimétrica,
  \item[$3.$]  cumple la identidad de Jacobi
 \begin{equation*}
   \pois{\pois{A}{B}}{C}+ \pois{\pois{B}{C}}{A} + \pois{\pois{C}{A}}{B}=0,
 \end{equation*}
 \item[$4.$] y cumple la regla de Leibniz
   \begin{equation*}
     \pois{A}{BC}=\pois{A}{B}C+B\pois{A}{C},
   \end{equation*}
   es decir, $\pois{A}{\bullet}$ es una derivación.
\end{enumerate}
  \end{prop}
En particular, las propiedades $1$-$3$ nos dicen que $(\mathscr{C}^{\infty}(M),\pois{ }{ })$ es un \emph{álgebra de Lie}. Al añadir la propiedad $4$ decimos que es un \emph{álgebra de Poisson}. 

  \begin{proof}
    La demostración de $1$ y $2$ es inmediata del hecho de que $\pois{F}{G}=\omega(X^F,X^G)$ y de la linealidad de $X^{\bullet}$.

    Para probar $3$ tenemos 
   \begin{align*}
     \lie{X^A}{X^B} (C) & = X^A\circ X^B (C) - X^B\circ X^A (C)  = X^A(\pois{B}{C})-X^B(\pois{A}{C}) \\
     &= \pois{A}{\pois{B}{C}}-\pois{B}{\pois{A}{C}}.
   \end{align*}
   Mientras que
   \begin{equation*}
     \lie{X^A}{X^B}(C)=X^{\pois{A}{B}}(C)=\pois{\pois{A}{B}}{C}.
   \end{equation*}
   Tenemos entonces
   \begin{equation*}
     \pois{\pois{A}{B}}{C}=\pois{A}{\pois{B}{C}}-\pois{B}{\pois{A}{C}}=-\pois{\pois{B}{C}}{A}-\pois{\pois{C}{A}}{B}.
   \end{equation*}
   Pasando el segundo término al otro lado obtenemos la identidad de Jacobi.
    
   La regla de Leibniz ($4$) es inmediata porque $\pois{A}{\bullet}(x)=X^A_x$ es una derivación.
  \end{proof}

  En general, si en una variedad diferenciable $M$ consideramos el conjunto de funciones $\mathscr{C}^{\infty}(M)$ y lo dotamos de la estructura de álgebra de Poisson mediante un corchete $\pois{ }{ }$, decimos que el par $(M,\pois{ }{ })$ es una \emph{variedad de Poisson}. Podemos reducir el enunciado de la proposición anterior simplemente a afirmar que \emph{toda variedad simpléctica es de Poisson}.

  \section{Mecánica en variedades simplécticas}
Una vez estudiadas las estructuras básicas de las variedades simplécticas, veamos cómo éstas plantean el marco natural en el que estudiar la mecánica clásica, dando lo que se denomina el \emph{formalismo hamiltoniano}. 

En general, un sistema mecánico se compone de tres partes: un conjunto de \emph{estados} o configuraciones, cuyos elementos contienen toda la información posible sobre el sistema en cierto instante, un conjunto de \emph{observables}, cantidades medibles del sistema, que «extraen» la información sobre éste, y una \emph{ley de evolución temporal}, que nos dice cómo se comportará el sistema a tiempos futuros. En el caso de la mecánica hamiltoniana, estas partes se pueden dar en el contexto de la geometría simpléctica, a saber, un \emph{sistema mecánico hamiltoniano} se compone de:

\paragraph{\textbf{Estados}} El conjunto de estados de un sistema hamiltoniano viene dado por una variedad simpléctica $(M,\omega)$, comúnmente llamada \emph{espacio de fases}.

\paragraph{\textbf{Observables}} Los observables del sistema son las funciones $\mathscr{C}^{\infty}(M)$, que además forman un álgebra de Poisson, como vimos en la sección anterior.

\paragraph{\textbf{Evolución temporal}} La ley de evolución temporal está determinada por una función $H:M\rightarrow \RR$ (característica de cada sistema), llamada \emph{hamiltoniano del sistema}. Las trayectorias del sistema seguirán las curvas integrales del campo $X^H$, luego, localmente, son solución de las ecuaciones de Hamilton
  \begin{equation*}
  \begin{cases}
     \dot{q_i}(t) = \parcial{H}{p_i}(t), \\
     \dot{p_i}(t) = -\parcial{H}{q_i} (t).
  \end{cases}
\end{equation*}
O, en su forma más compacta
  \begin{equation*}
    \begin{cases}
      \dot{q_i}(t)= \pois{H}{q_i(t)}, \\
      \dot{p_i}(t)= \pois{H}{p_i(t)}.
  \end{cases}
  \end{equation*}
  El flujo $\varphi^H$ generado por $X^H$ se llama \emph{flujo hamiltoniano} del sistema y contiene toda la información sobre su comportamiento. El hamiltoniano $H$ también determina la evolución temporal de los observables, según la \emph{ecuación de Liouville}
  \begin{equation*}
    \left.\frac{d}{dt}\right|_{t=0}F(\varphi_t^H(x))=\pois{H}{F}(x).
  \end{equation*}
  
  De manera más formal, podemos definir un \emph{sistema hamiltoniano} simplemente como un par $(M,H)$, donde $M=(M,\omega)$ es una variedad simpléctica y $H:M\rightarrow \RR$ una función.
  
  \paragraph{\bf Teorema de recurrencia de Poincaré}\mbox{}

  Como todos los campos hamiltonianos son simplécticos, los flujos hamiltonianos preservan la forma $\omega$ y en particular el volumen de Liouville $\omega^n$. Este resultado se conoce clásicamente como el \emph{teorema de Liouville}. A la vista de esto, los sistemas hamiltonianos tendrán una de las características más importantes de los sistemas que preservan el volumen, el \emph{teorema de recurrencia de Poincaré}\footnote{En inglés \textit{Poincaré recurrence theorem}. Nótese que aquí la palabra «recurrencia» no toma el significado habitual en matemáticas (por ejemplo en «construcción de sucesiones por recurrencia»), que se traduce del inglés \textit{recursion}. El diccionario Oxford define \textit{recursion} como \textit{``the repeated application of a recursive procedure or definition''}. Por otro lado, define \textit{recurrence} como \textit{``the fact of ocurring again''}, que podría traducirse también como «repetición» o «reaparición».}:
  \begin{prop}[Teorema de recurrencia de Poincaré]\label{tpoincare}
 Sea $D\subset \RR^n$ un conjunto de volumen finito y sea $g:D\rightarrow D$ que preserva el volumen euclídeo. Entonces, para cualquier $U\subset D$ abierto, hay un punto $x\in D$ y un $n\in \NN$, $n>0$, tal que $g^n(x) \in U$.
\end{prop}
\begin{proof}
  Consideramos la familia
    $\{g^nU | n \in \NN \}$.
  Todos estos conjuntos tienen el mismo volumen (mayor que 0, pues $U$ es abierto) y, si no se cortaran en ningún punto, $D$ tendría volumen infinito. Por tanto, existen $k,l\in \NN$, $k>l$ tal que
  \begin{equation*}
    g^kU\cap g^lU \neq \varnothing,
  \end{equation*}
  luego $A=g^{k-l}U \cap U \neq \varnothing$. Entonces, dado $y\in A$, existen $x \in U$ y $n=k-l$ tal que $y=g^n(x)\in U$.
\end{proof}
\begin{ejemplo}
  \em
  Sea $D$ una circunferencia y $g$ la rotación de ángulo $\alpha$. Si $\alpha=2\pi(m/n)$, entonces $g^n$ es la identidad y el resultado es obvio. Pero, si $\alpha$ es un múltiplo irracional de $2\pi$, entonces, por el teorema de recurrencia de Poincaré, para todo $x\in D$ y para todo $\delta>0$ existe un $n\in \NN$ tal que $g^n(x) \in B_{\delta}(x)$.
  De aquí se sigue que, dado $x \in D$, el conjunto $\{g^k(x)|k\in \NN\}$ es denso en $D$. Más adelante veremos una aplicación de este ejemplo a mecánica hamiltoniana.
\end{ejemplo}

\paragraph{\bf Transformaciones canónicas}\mbox{}

  Para terminar la sección, vamos a ver cómo se comportan las ecuaciones de Hamilton bajo difeomorfismos:
  \begin{prop}\label{cambiodif}
 Sea $M$ una variedad simpléctica y $H:M\rightarrow \RR$. Si $f:M\rightarrow M$ es un difeomorfismo, entonces 
 \begin{equation*}
   f^*i_{X^H}\omega=i_{X^{H\circ f}}\omega.
 \end{equation*}
\end{prop}
\begin{proof}
  Basta hacer los cálculos,
  \begin{equation*}
    f^*i_{X^H}\omega=f^*\dd H=\dd H \circ f_* = \dd (H\circ f)=i_{X^{H\circ f}}\omega.
  \end{equation*}
\end{proof}
Vamos a buscar entonces cambios de coordenadas que mantengan las ecuaciones de Hamilton invariantes, es decir, que al escribir las ecuaciones en las nuevas coordenadas se obtengan precisamente las ecuaciones de Hamilton del mismo hamiltoniano, expresado en las nuevas coordenadas.
\begin{defn}
  \em
  Sean $(M,H)$ un sistema hamiltoniano y $f:M\rightarrow M$ un difeomorfismo. Decimos que $f$ es una \emph{transformación canónica} del sistema si deja las ecuaciones de Hamilton invariantes, es decir, si $X^H=f_*X^{H\circ f}$.
\end{defn}

\begin{prop}\label{cambiohamilton}
  Sean $(M,\omega)$ una variedad simpléctica y $f:M\rightarrow M$ un difeomorfismo. Son equivalentes:
  \begin{enumerate}
    \item[$1$.] $f$ es un simplectomorfismo,
    \item[$2$.] para todo $H:M\rightarrow \RR$, $f$ es una transformación canónica del sistema $(M,H)$,
    \item[$3$.] $f$ deja invariante el corchete de Poisson, es decir, para cualesquiera $F,G\in \mathscr{C}^{\infty}(M)$, 
      \begin{equation*}
	\pois{F}{G}\circ f=\pois{F\circ f}{G\circ f}.
      \end{equation*}
  \end{enumerate}
\end{prop}

\begin{proof}

  $1\implies 2$. En primer lugar, por una comprobación inmediata se tiene que si $f$ es un difeomorfismo local, para cualquier forma $\lambda$, y para cualquier campo $X$, 
  \begin{equation*}
    f^*i_X \lambda = i_{(f_*)^{-1}X}f^* \lambda.
  \end{equation*}
  Como $f$ es un simplectomorfismo, $f^*\omega=\omega$, luego, si aplicamos la proposición \ref{cambiodif},
  \begin{equation*}
    i_{X^{H\circ f}}\omega=f^{*}(i_{X^H}\omega)=i_{(f_*)^{-1}X^H}f^*\omega=i_{(f_*)^{-1}X^H}\omega
  \end{equation*}
  por tanto $(f_*)^{-1}X^H=X^{H\circ f}$, de modo que $X^H=f_*X^{H\circ f}$, es decir, $f$ es una transformación canónica de $(M,H)$.

  $2\implies 1$. Si $f_*X^{H\circ f}=X^H$ para todo $H:M \rightarrow \RR$, entonces, de nuevo por la proposición \ref{cambiodif} y por la fórmula anterior,
  \begin{equation*}
    i_{X^H}\omega=i_{f_*X^{H\circ f}}\omega = (f^*)^{-1}(i_{X^{H\circ f}}f^*\omega)=(f^*)^{-1}f^*(i_{X^H}f^*\omega)=i_{X^H}f^*\omega.
  \end{equation*}
  Luego $f^* \omega = \omega$ y $f$ es un simplectomorfismo.

  $3\Longleftrightarrow 2$. Tenemos,
  \begin{align*}
    \pois{F}{G}\circ f(x)=X^G_{f(x)}(F)=\left( (f_{*})^{-1}X^G \right)_x(F\circ f)
  \end{align*}
  \begin{equation*}
    \pois{F\circ f}{G \circ f} = X^{G\circ f}(F \circ f).
  \end{equation*}

  Por tanto, $\pois{F}{G} \circ f=\pois{F\circ f}{G \circ f}$ para cualesquiera $F,G$ si y sólo si $f_*X^{G \circ f}= X^{G}$, para todo $G$.
\end{proof}

\section{Simetrías y leyes de conservación}
En esta sección damos la noción de \emph{cantidad conservada} o \emph{integral primera} de un sistema hamiltoniano y estudiamos la relación que tiene con las simetrías del sistema, mediante el mecanismo descubierto por Emmy Noether, que constituye una de las ideas centrales a toda la Física.
\begin{defn}
  \em
  Sea $(M,H)$ un sistema hamiltoniano. Una función $F:M \rightarrow \RR$ se dice que es una \emph{integral primera} del sistema o una \emph{cantidad conservada} si es constante a lo largo del flujo hamiltoniano. Esto es, si
  \begin{equation*}
    F(\varphi^H_t(x))=F(x)
  \end{equation*}
  para todo $t \geq 0$ y para todo $x \in M$.
\end{defn}

El primer ejemplo de integral primera es el propio hamiltoniano.
  \begin{prop}[Ley de conservación de la energía]
    $H$ es una integral primera del sistema hamiltoniano $(M,H)$.
  \end{prop}
  \begin{proof}
    Basta hallar la derivada de $H$ en la dirección de $X^H$ y ver que es 0. En efecto, dado $x \in M$,
    \begin{equation*}
      \frac{d}{dt}H\left( \varphi_t^H(x) \right)=d_{\varphi_t^H(x)}H(X_{\varphi_t(x)}^H) = \omega(X_{\varphi_t(x)}^H,X_{\varphi_t(x)}^H)=0,
    \end{equation*}
    ya que $d_xH=\omega(\bullet,X_x^H)$ y $\omega$ es antisimétrica.
  \end{proof}

Veamos ahora como el corchete de Poisson, ecuación de Liouville mediante, juega un papel central en todo este asunto. En efecto, de la ecuación de Liouville obtenemos inmediatamente lo siguiente:
\begin{prop}
  Una función $F:M \rightarrow \RR$ es una integral primera de $(M,H)$ si y sólo si $\pois{H}{F}$ (equivalentemente, $\pois{F}{H}$) es idénticamente nula.
\end{prop}

Además, si conocemos algunas integrales primeras, la identidad de Jacobi nos permite obtener otras nuevas:
\begin{prop}[Teorema de Poisson]
  Si $F_1, F_2$ son integrales primeras de $(M,H)$, entonces $\pois{F_1}{F_2}$ es también una integral primera de $(M,H)$.
\end{prop}
\begin{proof}
  Por la identidad de Jacobi,
  \begin{equation*}
    \pois{ \pois{F_1}{F_2}}{ H}= \pois{F_1}{ \pois{F_2}{H}} + \pois{F_2}{ \pois{H}{F_1} }=0,
  \end{equation*}
  ya que $F_1,F_2$ son integrales primeras.
\end{proof}

Veamos ahora el resultado principal de esta sección, que relaciona integrales primeras con simetrías.
  \begin{prop}[Teorema de Noether]
    Sea $(M,H)$ un sistema hamiltoniano y $F:M \rightarrow \RR$. Si $H$ es constante a lo largo del flujo de $X^F$, entonces $F$ es una integral primera de $(M,H)$. 
  \end{prop}
  \begin{proof}
    Como $H$ es constante a lo largo de $X^F$, es una integral primera de $(M,F)$, luego $\pois{H}{F}=\pois{F}{H}=0$ y $F$ es una integral primera de $(M,H)$. 
  \end{proof}
Otra forma de ver este mismo teorema es la siguiente:
\begin{prop}[Otra forma del teorema de Noether]
  Sea $(M,\omega)$ una variedad simpléctica conexa y sean $X^F$ y $X^G$ campos hamiltonianos en $M$. Los dos campos conmutan (y, por tanto, lo hacen los flujos que generan) si y sólo si $\pois{F}{G}$ es constante. 
\end{prop}
\begin{proof}
  Si $\pois{F}{G}=a \in \RR$, entonces
  \begin{equation*}
    \lie{X^F}{X^G}= X^{\pois{F}{G}} = X^a=0,
  \end{equation*}
  y recíprocamente
  \begin{equation*}
    \dd\pois{F}{G}=\omega(\bullet,X^{\pois{F}{G}})=\omega(\bullet,\lie{X^F}{X^G}).
  \end{equation*}
\end{proof}

Para entender bien qué son las «simetrías» del sistema consideremos la siguiente definición:
\begin{defn}
  \em
  Sea $(M,H)$ un sistema hamiltoniano y $G$ un grupo. Sea $\mathrm{Diff}(M)$ el grupo de difeomorfismos de $M$. Una \emph{$G$-simetría} del sistema $(M,H)$ es una acción
  \begin{align*}
    \varphi :G&\longrightarrow \mathrm{Diff}(M)\\ 
      g &\longmapsto \varphi_g, 
    \end{align*}
    tal que $H\circ \varphi_g = H$.
\end{defn}

Ahora, consideremos el caso en que $G$ sea un grupo de Lie y $\varphi$ una $G$-simetría diferenciable de un sistema hamiltoniano $(M,H)$. Si $g_t$ es un subgrupo uniparamétrico de $G$ entonces $g_t$ lleva asociado por la acción un flujo completo $\varphi_t=\varphi_{g_t}$. En el caso en que $\varphi_t$ sea un flujo generado por un hamiltoniano $F:M\rightarrow \RR$, entonces, por el teorema de Noether, $F$ es una cantidad conservada del sistema $(M,H)$. La mejor forma de ilustrar estas ideas es con ejemplos:

\begin{ejemplo}[Conservación del momento lineal]
  \em
  Consideremos un sistema hamiltoniano cuyo espacio de fases es el espacio simpléctico estándar $(\RR^{2n},\Omega_n)$. Decimos que el sistema es \emph{invariante por traslaciones} si la acción
  \begin{align*}
    \RR^n\times \RR^{2n}&\longrightarrow \RR^{2n}\\ 
      (y,(q,p)) &\longmapsto (q+y,p) 
    \end{align*}
del grupo $G=(\RR^n,+)$ sobre $\RR^{2n}$ es una $G$-simetría del sistema. Fijo ahora $y\in \RR^n$, si consideramos el flujo
\begin{equation*}
  \varphi_t(q,p)=(q+yt,p),
\end{equation*}
el hamiltoniano del sistema es invariante bajo este flujo. Busquemos ahora un posible hamiltoniano para $\varphi_t$. Para ello, hallemos su generador infinitesimal
\begin{equation*}
  X_y=\left.\frac{d}{dt}\right|_{t=0}\varphi_t(q,p)=(y,0)=\sum_{i=1}^n y_i \deriv{q_i}.
\end{equation*}
Ahora, si $F$ es el hamiltoniano asociado (suponiendo que exista) tenemos que
\begin{equation*}
  \dd F=-i_{X_y}\Omega_n=\sum_{i=1}^n(\dd p_i \wedge \dd q_i)\left(\bullet,\sum_{i=1}^n y_i\deriv{q_i} \right)=\sum_{i=1}^n y_i \dd p_i.
\end{equation*}
De modo que la cantidad conservada es $F=\sum_{i=1}^n y_i p_i=\esc{y}{p}$. Como esto es cierto para cualquier $y\in \RR^n$, en particular lo es para los vectores $e_i$ de la base canónica, de modo que, para cada $i=1,\dots,n$, $p_i$ es una cantidad conservada y, en general, el \emph{momento lineal} $p=(p_1,\dots,p_n)$ se conserva. De esta forma, hemos visto como la simetría bajo traslaciones lleva asociada la conservación del momento lineal.
\end{ejemplo}

\begin{ejemplo}[Conservación del momento angular]
  \em
  De nuevo, consideremos un sistema hamiltoniano cuyo espacio de fases es el espacio simpléctico estándar, en este caso 6 dimensional $(\RR^{3}\times \RR^3,\Omega_3)$. Decimos que el sistema es \emph{invariante por rotaciones} si la acción
  \begin{align*}
    \mathrm{SO}(3) \times \RR^{6}&\longrightarrow \RR^{6}\\ 
      (R,(q,p)) &\longmapsto (R(q),R(p)) 
    \end{align*}
    del grupo de rotaciones $G=\mathrm{SO}(3)$ con la operación de composición, es una $G$-simetría del sistema. Consideremos ahora el grupo uniparamétrico de rotaciones en torno al eje $OZ$, que matricialmente se representa como
    \begin{equation*}
      A(\phi)=\left(
      \begin{array}{ccc}
	\cos\phi & -\sin\phi & 0 \\
	\sin\phi & \cos\phi & 0 \\
	0 & 0 & 1 \\
      \end{array}\right).
    \end{equation*}
    Así, a este grupo uniparamétrico le podemos asociar el flujo
    \begin{equation*}
      \varphi_{\phi}(q,p)=(A(\phi)q,A(\phi)p),
    \end{equation*}
    cuyo generador infinitesimal será, por la regla de la cadena
    \begin{equation*}
      X=\left.\frac{d}{dt}\right|_{\phi=0}\varphi_{\phi}(q,p)=(\hat{A}q,\hat{A}p),
    \end{equation*}
    con 
    \begin{equation*}
      \hat{A}=\left.\frac{d}{dt}\right|_{\phi=0}A(\phi)=\left(
      \begin{array}{ccc}
	0 & -1 & 0 \\
	1 & 0 & 0 \\
	0 & 0 & 0 \\
      \end{array}\right).
    \end{equation*}
    De modo que 
    \begin{equation*}
      X=-q_2\deriv{q_1}+q_1\deriv{q_2}-p_2\deriv{p_1}+p_1\deriv{p_2}.
    \end{equation*}
    Luego,
    \begin{equation*}
      -i_X\Omega_3=-q_2\dd p_1+q_1\dd p_2 +p_2 \dd q_1 - p_1 \dd q_2.
    \end{equation*}

    Definimos ahora el \emph{momento angular} como la función $L:\RR^6 \rightarrow \RR^3$ dada por el producto vectorial $L(q,p)=q\times p$, de modo que su componente $k$-ésima es $L_k(q,p)=\sum_{i,j=1}^3 \epsilon_{ijk}q_ip_j$, donde $\epsilon_{ijk}$ es la paridad de $(i,j,k)$ como permutación de $(1,2,3)$. Ahora, 
    \begin{equation*}
      \dd L_3 = \dd (q_1p_2 - q_2p_1)=-q_2 \dd p_1 + q_1 \dd p_2 + p_2 \dd q_1 - p_1 \dd q_2 = -i_X\Omega_3.
    \end{equation*}

    De modo que, aplicando el teorema de Noether, la simetría del sistema bajo las rotaciones en torno al eje $OZ$ lleva asociada una conservación de la tercera componente del momento angular $L_3$. Podemos hacer un cálculo análogo para los casos de las rotaciones en torno a los ejes $OX$ y $OY$, de forma que obtendríamos la conservación de las componentes $L_1$ y $L_2$. Finalmente, llegaríamos a que la simetría bajo rotaciones lleva asociada la conservación del momento angular.
\end{ejemplo}
