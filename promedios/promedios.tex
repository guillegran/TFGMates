\section{Un poco más de movimiento condicionalmente periódico}\label{7}
Como colofón, una vez tenemos a nuestra disposición la teoría de Arnold-Liouville, sabemos que el flujo en los toros invariantes de los sistemas integrables en el sentido de Liouville será condicionalmente periódico. En esta sección definiremos bien qué significa esto y obtendremos un teorema muy importante sobre este tipo de sistemas.

\begin{defn}[Movimiento condicionalmente periódico en $\TT^n$]
  Sean $\TT^n$ el toro n-dimensional y $\phi=(\phi_1,\dots,\phi_n)$ coordenadas angulares. Se entiende por un \emph{movimiento condicionalmente periódico} en el toro el flujo uniparamétrico dado por 
  \begin{equation*}
    \phi(t)=\phi(0)+\omega t
  \end{equation*}
  con $\omega=(\omega_1,\dots,\omega_n)$ \emph{frecuencias} constantes en el toro. Las frecuencias $\omega$ se dicen \emph{independientes} si, sea $k \in \ZZ^n$, entonces $\esc{k}{\omega}=0$ si y sólo si $k=0$.
\end{defn}
\begin{defn}[Promedios espacial y temporal]
  Sea $f:\TT^n \rightarrow \RR$ una función integrable Riemann,
  \begin{enumerate}
    \item El \emph{promedio espacial} de $f$ en $\TT^n$ es el número
      \begin{equation*}
	\bar{f}=\frac{1}{(2\pi)^n}\int_0^{2\pi} \cdots \int_0^{2\pi} f(\phi) d\phi_1,\dots,d\phi_n.
      \end{equation*}
    \item El \emph{promedio temporal} de $f$ en $\TT^n$ es la función
      \begin{equation*}
	f^*(\phi_0)=\lim_{T\rightarrow \infty} \int_0^{T}f(\phi_0+\omega t) dt,
      \end{equation*}
      definida en los puntos $\phi_0$ en los que exista el límite.
  \end{enumerate}
\end{defn}

\begin{thm}[Teorema de los promedios]
  Si $f:\TT^n \rightarrow \RR$ es una función integrable Riemann y las frecuencias $\omega$ son independientes, el promedio temporal está bien definido en todo el toro $\TT^n$ y coincide en todo punto con el promedio espacial.
\end{thm}
\begin{proof}
 Daremos la demostración en varios pasos:
 \begin{enumerate}
   \item Consideramos funciones de la forma $e^{i\esc{k}{\phi}}$, $k\in \ZZ^n$. Si $k=0$, entonces $\bar{f}=f=f^*=1$. Si $k\neq0$, $\bar{f}$ es una integral a periodos en funciones trigonométricas, luego es igual a 0. Por otra parte
     \begin{equation*}
       \int_0^T e^{i\esc{k}{\phi_0 + \omega t}} dt= e^{i\esc{k}{\phi_0}}\int_0^T e^{i\esc{k}{\omega}t}dt=e^{i\esc{k}{\phi_0}}\frac{e^{i\esc{k}{\omega}T}-1}{i \esc{k}{\omega}}.
     \end{equation*}
     Por tanto, el promedio temporal será
     \begin{equation*}
       \lim_{T\rightarrow \infty}\frac{e^{i\esc{k}{\phi_0}}}{i \esc{k}{\omega}}\frac{e^{i\esc{k}{\omega}T}-1}{T}=0.
     \end{equation*}
   \item Como los promedios dependen linealmente de $f$, también coincidirán para los polinomios trigonométricos
     \begin{equation*}
       f=\sum_{|k|<N}f_ke^{i\esc{k}{\omega}}.
     \end{equation*}
   \item Dado $\varepsilon >0$, si $f$ es continua y real por el teorema de Weierstrass podemos aproximarla por un polinomio trigonométrico $P$ que cumpla $|f-P|<\frac{1}{2}\varepsilon$. Sean $P_1=P-\frac{1}{2}\varepsilon$, $P_2=P+\frac{1}{2}\varepsilon$ entonces
     \begin{equation*}
       \bar{P_2}-\bar{P_1}=\frac{1}{(2\pi)^n}\int_{\TT^n} (P_2 - P_1) d\phi = \frac{1}{(2\pi)^n}\varepsilon (2\pi)^n= \varepsilon.
     \end{equation*}
   \item Dado $\varepsilon >0$, si $f$ es real e integrable Riemann, entonces existen dos funciones continuas $f_1,f_2$ tales que $f_1<f<f_2$ y $\int_{\TT^n}\frac{1}{(2\pi)^n}(f_2-f_1)d\phi<\frac{1}{3}\varepsilon$. Tomando ahora $P_1,P_2$ polinomios trigonométricos tales que $P_1<f_1<f_2<P_2$ y $\int_{\TT^n}\frac{1}{(2\pi)^n}(P_i-f_i)d\phi < \frac{1}{3}\varepsilon$, para $i=1,2$, entonces
     \begin{equation*}
       \bar{P_2}-\bar{P_1}=\frac{1}{(2\pi)^n}\int_{\TT^n} (P_2 - P_1) d\phi = \frac{1}{(2\pi)^n}\varepsilon (2\pi)^n= \varepsilon.
     \end{equation*}
   \item Por último, sea $\varepsilon>0$, entonces existen dos polinomios trigonométricos $P_1,P_2$ tales que $P_1<f<P_2$ y $\bar{P_2}-\bar{P_1}<\varepsilon$. Ahora, como $f<P_2$,
     \begin{equation*}
       \frac{1}{T}\int_0^T f(\phi(t))dt < \frac{1}{T}\int_0^T P_2(\phi(t))dt,
     \end{equation*}
     luego
     \begin{equation*}
       \left| \frac{1}{T}\int_0^T f(\phi(t))dt- \bar{f} \right| <\left| \frac{1}{T}\int_0^T P_2(\phi(t))dt- \bar{f} \right| < \left| \frac{1}{T}\int_0^T P_2(\phi(t))dt- \bar{P_2}\right| + |\bar{P_2}-\bar{f}|.
     \end{equation*}
     Pero, como $P_1<f<P_2$, por la monotonía de la integral $\bar{P_1}<f<\bar{P_2}$, luego $|\bar{P_2}-\bar{f}|<|\bar{P_2}-\bar{P_1}|<\varepsilon$. Además, como $P_2$ es un polinomio trigonométrico existe un $T_0$ tal que, si $T>T_0$
     \begin{equation*}
       \left| \frac{1}{T}\int_0^T P_2(\phi(t))dt- \bar{P_2}\right| < \varepsilon .
     \end{equation*}
     Finalmente, obtenemos lo que queríamos probar
     \begin{equation*}
       \left| \frac{1}{T}\int_0^T f(\phi(t))dt- \bar{f} \right| <\left| \frac{1}{T}\int_0^T P_2(\phi(t))dt- \bar{P_2}\right| + |\bar{P_2}-\bar{f}|<\varepsilon + \varepsilon = 2\varepsilon,
     \end{equation*}
     luego $f^*(\phi_0)=\lim_{t\rightarrow \infty}\frac{1}{T}\int_0^T f(\phi(t)) dt = \bar{f}$.
 \end{enumerate}
\end{proof}
\begin{corol}
  Si las frecuencias son independientes, entonces, para todo $\phi_0 \in \TT^n$, \[\{\phi(t)=\phi_0+\omega t|t\in \RR\}\] es denso en el toro $\TT^n$.
\end{corol}
\begin{proof}
  En caso contrario, sea un abierto $D$ del toro que no tiene ningún punto de la trayectoria $\phi(t)$. Construimos la función 
  \begin{equation*}
    f(\phi)=\left\lbrace
    \begin{array}{ll}
      0 & \text{si } \phi \not\in D \\
      \frac{(2\pi)^n}{\int_D d\phi} & \text{si } \phi \in D.
    \end{array}
    \right.
  \end{equation*}
  Claramente, $\bar{f}=1$, pero $f^*(\phi_0)=0$, lo que contradice el teorema de los promedios.
\end{proof}
\begin{corol}
  Sea $D\subset \TT^n$ un conjunto medible Jordan. Sea $A_D=\{t\in \RR | \phi(t) \in D\}$ (que también es medible Jordan) y sea $\tau_D(T)=\int_0^T\chi_{A_D}(t)dt$. Entonces
  \begin{equation*}
    \lim_{T\rightarrow \infty}\frac{\tau_D(T)}{T}= \frac{\mathrm{Vol}(D)}{(2\pi)^n}.
  \end{equation*}
\end{corol}
\begin{proof}
  Aplicamos el teorema a $\chi_D$, entonces $\int_0^T \chi_D(\phi(t))dt=\int_0^T \chi_{A_D}(t)dt=\tau_D(t)$ y $\bar{\chi}_D=(2\pi)^{-n}\mathrm{Vol}(D)$. Finalmente, por el teorema de los promedios
  \begin{equation*}
    \bar{\chi}_D=\frac{\mathrm{Vol}(D)}{(2\pi)^n}=\lim_{T\rightarrow \infty}\frac{1}{T}\int_0^T \chi_D(\phi(t))dt=\lim_{T\rightarrow \infty}\frac{\tau_D(T)}{T}.
  \end{equation*}
\end{proof}
