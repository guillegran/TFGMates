
\section{Espacios vectoriales simplécticos}\label{sec:vectoriales}
En esta sección repasamos algunos conceptos de álgebra lineal necesarios para estudiar geometría simpléctica. Introduciremos la noción de \emph{espacio vectorial simpléctico} y veremos algunas de sus propiedades. Muchos de los resultados aquí expuestos pueden leerse más desarrollados en \cite{algebra}.
\begin{defn}[Espacio vectorial simpléctico]
  \em
Un \emph{espacio vectorial simpléctico} es un par ordenado $(V,\omega)$, donde $V$ es un espacio vectorial sobre $\RR$ y 
	\[
	  \omega: V \times V \rightarrow \RR
	\]
	es una forma bilineal antisimétrica no degenerada.
\end{defn}

\paragraph{\bf Clasificación de formas bilineales antisimétricas}
  \begin{enumerate}
    \item Sean $\omega$ una forma bilineal antisimétrica sobre un espacio vectorial $V$ de dimensión finita y $M$ su matriz asociada. Entonces existe $n \leq \frac{1}{2}\dim(V)$ tal que $M$ es congruente con
      \[
	\left(
	\begin{array}{ccc}
	  0 & I_n & 0 \\
	  -I_n & 0 & 0 \\
	  0 & 0 & 0
	\end{array}\right),
      \]
      donde $I_n$ es la matriz identidad $n \times n$.
    \item Si $(V,\omega)$ es un espacio vectorial simpléctico de dimensión finita, entonces $\dim(V)=2n$ para cierto $n \in \NN$.
    \item Sea $(V,\omega)$ un espacio vectorial simpléctico. Se dice que $\mathcal{B}\subset V$ es una \emph{base simpléctica} de $V$ si la matriz asociada a $\omega$ en $\mathcal{B}$ es
\[
  J_n :=
\left(
	\begin{array}{cc}
	  0 & I_n  \\
	  -I_n & 0 
	\end{array}\right).
      \]
    \item Podemos ver la forma bilineal $\omega$ de matriz asociada $J_n$ como una $2$-forma alternada en $V$. Si $\{u_1,\dots,u_n,v_1,\dots,v_n\}$ es una base simpléctica y $\{\varphi_1,\dots, \varphi_n, \psi_1,\dots, \psi_n \}$ es su base dual, entonces
  \[
    \omega = \sum_{i=1}^n \left(\sum_{\sigma \in \mathfrak{S}_2} (-1)^{\sigma} (\psi_i \otimes \varphi_i)^{\sigma}\right)= \sum_{i=1}^n \psi_i \wedge \varphi_i.
  \]
  \item Llamaremos \emph{$n$-ésima forma simpléctica estándar} a la forma $\omega_n:\RR^{2n} \rightarrow \RR^{2n}$ de matriz asociada $J_n$ en la base canónica de $\RR^{2n}$. Llamaremos \emph{espacio simpléctico estándar $2n$-dimensional} a $(\RR^{2n},\omega_n)$. Deducimos también de lo anterior que todo espacio vectorial simpléctico de dimensión $2n$ es isomorfo a $(\RR^{2n},\omega_n)$.
\end{enumerate}
\paragraph{\bf Aplicaciones simplécticas}\mbox{}

  Sean $(V,\omega)$ y $(V',\omega')$ espacios vectoriales simplécticos, decimos que una aplicación lineal $f:V \rightarrow V'$ es \emph{simpléctica} si
  \[
    \omega'(f(v),f(w)) = \omega(v,w)
  \]
  para cualesquiera $v,w \in V$.

Ahora, si $w\neq 0$ y $f(v)=0$, entonces $\omega(v,w)=0$ y, como $\omega$ es no degenerada, $v=0$, de modo que $\ker f=\{0\}$. Es decir, toda aplicación simpléctica es inyectiva.

En el caso en que $V$ y $V'$ tienen la misma dimensión, entonces $f$ es un isomorfismo que lleva una base simpléctica de $(V,\omega)$ a una base simpléctica de $(V',\omega')$. 

\paragraph{\bf El grupo simpléctico}\mbox{}

  Sea $(V,\omega)$ un espacio vectorial simpléctico de dimensión $2n$. El conjunto de las aplicaciones lineales simplécticas de $V$ en $V$ es un subgrupo de $\mathrm{GL}(2n)$. Este grupo se conoce como \emph{$n$-ésimo grupo simpléctico} y se denota por $\mathrm{Sp}(n)$.

  Asignando a cada $f\in \mathrm{Sp}(n)$ su matriz asociada $A$ en cierta base, el grupo simpléctico se puede representar por medio de las matrices $2n\times 2n$ que cumplen
  \begin{equation*}
    J_n=A^tJ_n A.
  \end{equation*}
  Estas matrices se dicen \emph{matrices simplécticas}.
  De aquí deducimos inmediatamente, por un razonamiento análogo al que haríamos para matrices ortogonales, que todas las matrices simplécticas tienen determinante $1$ o $-1$. Sin embargo, podemos probar un resultado aún más fuerte sobre el grupo simpléctico: su \emph{subgrupo especial} (es decir, el subgrupo de las aplicaciones de determinante $1$) coincide con él mismo.
\begin{prop}
  Toda aplicación lineal simpléctica tiene determinante $1$.
\end{prop}
\begin{proof}
  Sea $\omega_n$ la forma simpléctica canónica, $\Omega=\omega_n \wedge \overset{(n)}{\cdots} \wedge \omega_n$ es una forma de grado máximo, luego, por el teorema del determinante, $f^*(\Omega)=\det(f) \Omega$. Ahora, como $f$ es simpléctica, 
  \begin{equation*}
    f^*(\Omega) = (\omega_n \circ f) \wedge \overset{(n)}{\cdots} \wedge (\omega_n \circ f)=\omega_n \wedge \overset{(n)}{\cdots} \wedge \omega_n = \Omega.
  \end{equation*}
  Por tanto, $\det(f)=1$.
\end{proof}
\section{Variedades simplécticas y simplectomorfismos}\label{sec:variedades}
Una vez repasados los conceptos básicos de la geometría lineal simpléctica, estamos preparados para entender lo que son las \emph{variedades simplécticas} y estudiar las propiedades que comparten.
\begin{defn}
  \em
  Una \emph{variedad simpléctica} es un par ordenado $(M,\omega)$ donde $M$ es una variedad diferenciable de dimensión $2n$ y $\omega \in \Gamma^2(M)$ es cerrada y no degenerada (para todo $x \in M$, $\omega_x$ es no degenerada). 
\end{defn}
\begin{defn}
  \em
  Sea $(M,\omega)$ una variedad simpléctica. Una \emph{carta simpléctica en un punto $x \in M$} es un par $(U,(q,p))$, donde $U\subset M$ es un entorno de $x$ y $(q,p)=(q_1,\dots,q_n,p_1,\dots,p_n)$ es un sistema de coordenadas en $x$ tal que:
\begin{equation*}
  \omega= \sum_{i=1}^n \dd p_i \wedge \dd q_i.
\end{equation*}
\end{defn}

Vamos a probar ahora uno de los resultados fundamentales de la geometría simpléctica, que confirma una de las intuiciones que podríamos tener: toda variedad simpléctica es localmente como un espacio vectorial simpléctico.

\begin{thm}[Darboux]
  Sea $(M,\omega)$ una variedad simpléctica. Entonces para cada $x\in M$ hay una carta simpléctica en $x$.
\end{thm}
\begin{proof}
  El procedimiento a seguir para demostrar el teorema es el siguiente:
  \begin{enumerate}
    \item Vamos a encontrar un entorno $U \subset M$ y un difeomorfismo $\varphi : U \rightarrow \RR^{2n}$ tal que $\omega_1= (\varphi^*)^{-1}\omega$ es una forma bilineal antisimétrica con coeficientes constantes en $\RR^{2n}$.
    \item Tomamos $\psi: \RR^{2n} \rightarrow \RR^{2n}$ un cambio de base tal que la matriz asociada a $\omega_1$ en la nueva base es $J_n$. 
    \item Tomamos como sistema de coordenadas $(q,p)=\psi \circ \varphi$, entonces $(U,(q,p))$ es una carta simpléctica en $x$.
  \end{enumerate}

\begin{center}
\begin{tikzpicture}
\node (A) at (-2.5,2) {$U$};
\node (B) at (-1,2) {$\RR^{2n}$};
\node (C) at (-1,0) {$\RR^{2n}$};
\node (D) at (1,2) {$ \omega$};
\node (E) at (3,2) {$\omega_1$};
\node (F) at (3,0) {$\omega_n$};
\path[->,font=\scriptsize, >=angle 90]
(A) edge node[above]{$\varphi$} (B)
(A) edge node[left]{$(q,p)$} (C)
(B) edge node[right]{$\psi$} (C);
\path[<-|,font=\scriptsize, >=angle 90]
(D) edge node[left] {$ (q,p) ^*$} (F)
(D) edge node[above] {$\varphi^*$} (E)
(E) edge node[right] {$\psi^*$} (F);
\end{tikzpicture}
\end{center}

  Falta hallar $\omega_1$ y $\varphi$. Sea $\omega_1=\omega_x$, que tiene coeficientes constantes, y sea
  \begin{equation*}
    \omega_t=t\omega_1 + (1-t) \omega = \omega + t(\omega_1-\omega),
  \end{equation*}
  para cada $t\in \RR$. Como $\omega_{t,x}=\omega_x$, para todo $t \in \RR$, $\omega_t$ es no degenerada en $x$. Por tanto, sea $J$ intervalo abierto acotado tal que $[0,1]\in J$, existe un entorno $U \subset M$ de $x$ en el que $\omega_t$ es no degenerada para cada $t\in \bar{J}$.

  Ahora, como $\omega_1-\omega$ es cerrada, podemos tomar $U$ tal que sea difeomorfo a $\RR^{2n}$, luego existirá una 1-forma $\alpha$ en $U$ tal que $\omega_1-\omega=\dd \alpha$. Además, como $\alpha$ está definida salvo una constante, podemos asumir $\alpha_x=0$.

  Para $t \in J$, sea $X_t$ el campo dependiente del tiempo tal que $i_{X^t}\omega_t= - \alpha$ y con $X_{t,x}=0$. Ahora, si $\varphi:V\rightarrow U$ es el flujo dependiente del tiempo asociado a $X$, entonces $\varphi(t,0,x)=x$ para todo $t\in J$, luego $J\times \left\{ 0 \right\}\times \left\{ x \right\} \subset V$. Como $V$ es abierto en $J\times J\times M$ y $[0,1]$ es compacto, existe un entorno $U_0$ de $x$  tal que $[0,1]\times \left\{ 0 \right\}\times U_0 \subset V$. 
  
  Por tanto, si $\varphi_t = \varphi(t,0,\bullet)$ para cada $t\in [0,1]$ se sigue
  \begin{align*}
    \frac{d}{dt}(\varphi_t^*\omega_t) & = \varphi_t^* (L_{X_t}\omega_t) + \varphi_t^* \left( \frac{d}{dt}\omega_t \right) = \varphi_t^* (i_{X_t}(\dd \omega)+\dd (i_{X_t}\omega))+\varphi_t^*(\omega_1-\omega) \\
    & = \varphi_t^*(0-\dd \alpha + \omega_1 - \omega) = 0,
  \end{align*}
  donde hemos usado la fórmula de Cartan. De modo que $\varphi_1^*\omega_1=\varphi_0^*\omega_0=\omega$. Además, como $\varphi_1^*$ es simpléctica, es un isomorfismo, lo que implica que $\varphi_1$ es un difeomorfismo local. Así, reduciendo el entorno si es necesario, encontramos la carta que estábamos buscando.
\end{proof}

Vamos a dar ahora dos propiedades fuertes de las variedades simplécticas:
\begin{prop}
  Toda variedad simpléctica es orientable.
\end{prop}
\begin{proof}
  Como $\omega$ es no degenerada, $\omega^n=\omega \wedge \overset{(n)}{\cdots} \wedge \omega$ es una forma de grado máximo nunca nula, luego la variedad es orientable.
\end{proof}

A esta $\omega^n$ se le suele llamar \emph{volumen de Liouville} de $M$ ya que, aunque aquí no lo desarrollemos, está relacionada con el elemento de volumen en $M$ si definieramos en ésta una estructura riemanniana.

\begin{prop}
  Si $(M,\omega)$ es una variedad simpléctica compacta y sin borde, entonces su segundo grupo de cohomología de de Rham es no trivial: $\omega$ no es exacta.
\end{prop}
\begin{proof}
  Consideramos en $M$ el volumen de Liouville $\omega^n$. Ahora, si $\omega=\dd \alpha$, entonces $\omega^n=\dd (\alpha \wedge \omega^{n-1})$. Por el teorema de Stokes
  \begin{equation*}
    \int_M \omega^n=\int_M \dd (\alpha \wedge \omega^{n-1}) = \int_{\partial M=\varnothing} \alpha \wedge \omega^{n-1}=0.
  \end{equation*}
  Pero $M$ es compacta y $\omega^n$ es una forma diferencial de grado máximo, luego $\int_M \omega^n$ no puede ser igual a $0$. Por tanto, $\omega$ es cerrada y no es exacta, luego $[\omega]\neq 0$ y $H^2(M) \neq \{0\}$.
\end{proof}

\begin{ejemplo}
  \em
  Podemos dar un ejemplo de variedad simpléctica si consideramos la esfera $\SF^2$ y su 2-forma diferencial $\omega$ nunca nula que sabemos que tiene por ser orientable. 
\end{ejemplo}

Como consecuencia del resultado anterior, las esferas de dimensión superior a $2$ no admiten una estructura simpléctica. En efecto, tenemos el siguiente teorema:

\begin{thm}
  Si $n>1$, entonces $H^2(\SF^n)={0}$.
\end{thm}

\begin{proof}
  Sean $\omega$ una 2-forma en $\SF^n$, $n>2$, $x\in \SF^n$ y $U$ un disco entorno de $x$. Entonces existe una 1-forma $\alpha$ definida en $U$ tal que $\omega=\dd \alpha$ en $U$. Sea $\theta$ una función meseta que valga 1 en un entorno $V\subset U$ de $x$ relativamente compacto de $U$ y $0$ fuera de $U$, la forma 
  \begin{equation*}
    \omega_1=\omega - \dd(\theta \alpha)
  \end{equation*}
  es cerrada con soporte compacto en $\SF^n \backslash \{x\}$, que es difeomorfa, por proyección estereográfica $\varphi$, a $\RR^n$. Por tanto, existe una 1-forma $\beta$ de $\RR^n$ tal que $\dd \beta = \varphi^* \omega_1$. Con todo esto, tenemos
  \begin{equation*}
    \omega=\dd\left(\theta \alpha + \left( \varphi^{-1} \right)^* \beta \right) ,
  \end{equation*}
  luego $\omega$ es exacta en $\SF^n$ y 
  \begin{equation*}
    H^2(\SF^n)= \{0\}.
  \end{equation*}
\end{proof}

Para terminar la sección, vamos a estudiar los morfismos en la categoría de las variedades simplécticas:
\begin{defn}
  \em
  Sean $(M,\omega)$ y $(M',\omega')$ variedades simplécticas y $f:M \rightarrow M'$ una aplicación diferenciable. Decimos que $f$ es un \emph{simplectomorfismo} o una \emph{transformación canónica} \footnote{Esta nomenclatura es más común en Física} si
  \begin{equation*}
    f^* \omega' = \omega,
  \end{equation*}
  es decir, si para cada $x\in M$, $(d_xf)^*\omega'_{f(x)}=\omega_x$. Equivalentemente, $f$ es simpléctica si la aplicación lineal $d_xf:T_xM \rightarrow T_{f(x)}M'$ es simpléctica.
\end{defn}
\begin{prop}
  Sean $(M,\omega)$ y $(M',\omega')$ variedades simplécticas y $f:M \rightarrow M'$ una aplicación diferenciable. Entonces $f$ es un simplectomorfismo si y sólo si, para cada $x \in M$, si $(U,(q,p))$ es una carta simpléctica en $x$, entonces $(f(U),(q\circ f, p \circ f))$ es una carta simpléctica en $f(x)$.
\end{prop}
\begin{proof}\leavevmode
Basta ver que
\begin{equation*}
  \sum_{i=1}^n \dd (p_i \circ f) \wedge \dd (q_i \circ f) = \sum_{i=1}^n f^* (\dd p_i \wedge \dd q_i) = f^* \left( \sum_{i=1}^n \dd p_i \wedge \dd q_i \right) = f^* \omega
\end{equation*}
es igual a $\omega$ si y sólo si $f$ es un simplectomorfismo.

\end{proof}

Enunciamos también una consecuencia inmediata de la definición de simplectomorfismo:
\begin{prop}\label{volumen}
  Los simplectomorfismos preservan la forma de volumen $\omega^n$.
\end{prop}

\section{Campos y flujos hamiltonianos}\label{sec:hamilton}
Sea $M$ una variedad simpléctica\footnote{A partir de este momento denotaremos las variedades simplécticas $(M,\omega)$ simplemente por $M$}. Análogamente a lo visto en la fórmula \eqref{isomorfismo}, la aplicación 
  \begin{equation*}
    \begin{array}{rcl}
      T_xM & \longrightarrow & (T_xM)^* \\
      \xi & \longmapsto & \omega(\xi,\bullet),
      \end{array} 
  \end{equation*}
es un isomorfismo lineal entre campos y 1-formas. Resulta que a cada campo $X$ le podemos asignar la forma $i_X \omega$. Veremos ahora como esto nos permite asociar campos a las funciones de la variedad.
\begin{defn}
  \em
  Dada una función $H:M \rightarrow \RR$, se llama \emph{campo hamiltoniano generado por H} al campo $X^H$ tal que $dH=i_{X^H}\omega=\omega(X^H,\bullet)$.
  Se llama \emph{flujo hamiltoniano generado por H} y se denota por $\varphi^H$ al flujo de $X^H$.
\end{defn}
  Obtenemos ahora una versión geométrica e independiente de coordenadas de las clásicas ecuaciones de Hamilton
  \begin{equation*}
    X^H_x= \left.\frac{d}{dt}\right| _{t=0}\varphi^H_t(x).
  \end{equation*}
Usando este formalismo podemos dar una formulación canónica de la mecánica clásica.
\begin{defn}
  \em
  Un \emph{sistema mecánico hamiltoniano} es un par $(M,H)$, donde $M$ es una variedad simpléctica (comúnmente llamada \emph{espacio de fases}) y $H:M \rightarrow \RR$ es una función (normalmente llamada \emph{hamiltoniano} del sistema).

  Se llama \emph{flujo hamiltoniano del sistema} o \emph{evolución temporal del sistema} al flujo hamiltoniano generado por $H$.
\end{defn}

\begin{ejemplo}
  \em
  Consideremos el espacio simpléctico canónico $(\RR^{2n},\omega_n)$, con un hamiltoniano $H:\RR^{2n}\rightarrow \RR$. Podemos escribir un punto $x\in \RR^{2n}$ en coordenadas simplécticas $x = (q_1,\dots,q_n,p_1,\dots,p_n)$ y el hamiltoniano $H(q_1,\dots,q_n,p_1,\dots,p_n)$. Podemos escribir entonces
  \begin{equation*}
    \dd H = \sum_{i=1}^n \parcial{H}{q_i} \dd{q_i} + \parcial{H}{p_i} \dd{p_i}
  \end{equation*}
  y el campo hamiltoniano tendrá la forma
  \begin{equation*}
    X^H = \sum_{i=1}^n X_{q_i}^H \deriv{q_i} + X_{p_i}^H \deriv{p_i}.
  \end{equation*}
  Ahora, 
  \begin{equation*}
    \dd H = i_{X^H}\omega_n = \sum_{i=1}^n \dd p_i \wedge \dd q_i \left(X_{q_i}^H \deriv{q_i} + X^H_{p_i} \deriv{p_i},\bullet\right)  = \sum_{i=1}^n X_{q_i}^H \dd p_i -X_{p_i}^H \dd q_i.
  \end{equation*}
  Por tanto, las componentes del campo hamiltoniano son $X^H_{q_i}=\parcial{H}{p_i}$ y $X^H_{p_i}=-\parcial{H}{q_i}$.

  El flujo hamiltoniano tendrá la forma $\varphi^H_t(x) = (q_1(t),\dots,q_n(t),p_1(t),\dots,p_n(t))$. Luego su generador infinitesimal será de la forma $X^H_x=(\dot{q_1}(t),\dots,\dot{q_n}(t),\dot{p_1}(t),\dots,\dot{p_n}(t))$. Obtenemos entonces las ecuaciones de Hamilton en su forma clásica \eqref{hamilton}:
  \begin{equation*}
  \begin{cases}
     \dot{q_i}(t) & = \parcial{H}{p_i}(t) \\
     \dot{p_i}(t) & = -\parcial{H}{q_i} (t),
  \end{cases}
\end{equation*}
  para $i=1,\dots,n$.
\end{ejemplo}

Vamos a ver ahora como se comportan los campos hamiltonianos bajo la acción de funciones diferenciables.

\begin{prop}
 Sean $M,M'$ variedades simplécticas y $H:M\rightarrow \RR$. Si $f:M\rightarrow M'$ es una aplicación diferenciable, entonces 
 \begin{equation*}
   f^*i_{X^H}\omega=i_{X^{H\circ f}}\omega'.
 \end{equation*}
\end{prop}
\begin{proof}
  Basta hacer los cálculos,
  \begin{equation*}
    f^*i_{X^H}\omega=f^*\dd H=\dd H \circ f_* = \dd (H\circ f)=i_{X^{H\circ f}}\omega.
  \end{equation*}
\end{proof}

Como consecuencia tenemos que los simplectomorfismos preservan las ecuaciones de Hamilton.

\begin{prop}
  Sean $M$ una variedad simpléctica y $f:M\rightarrow M$. La función $f$ es un simplectomorfismo si y sólo si para todo $H:M\rightarrow \RR$
  \begin{equation*}
    f_*X^H=X^{H\circ f^{-1}}.
  \end{equation*}
\end{prop}
\begin{proof}
  En primer lugar, por una comprobación inmediata se tiene, para cualquier difeomorfismo $f$, para cualquier forma $\lambda$, y para cualquier campo $X$, 
  \begin{equation*}
    (f^{-1})^*i_X \lambda = i_{f_*X}(f^{-1})^* \lambda.
  \end{equation*}
  Ahora,
  \begin{equation*}
    (f^{-1})^*i_{X^H}\omega=i_{X^{H \circ f^{-1}}}\omega,
  \end{equation*}
  mientras que, por lo anterior,
  \begin{equation*}
    (f^{-1})^*(i_{X^H}\omega)=i_{f_*X^H} (f^{-1})^*\omega.
  \end{equation*}
  Como $f$ es un simplectomorfismo, $f^{-1}$ también lo será, luego $(f^{-1})^* \omega = \omega$.
  Tenemos entonces,
  \begin{equation*}
    i_{X^{H \circ f^{-1}}} \omega = i_{f_*X^H}\omega,
  \end{equation*}
  es decir 
  \begin{equation*}
    \omega(X^{H\circ f^{-1}},\bullet)=\omega(f_*X^H,\bullet)
  \end{equation*}
  y, como $\omega$ es no degenerada,
  \begin{equation*}
    X^{H \circ f^{-1}}=f_*X^H.
  \end{equation*}

  Recíprocamente, si $f_*(X^H)=X^{H \circ f^{-1}}$ para todo $H:M \rightarrow \RR$, entonces
  \begin{equation*}
    i_{X^{H\circ f^{-1}}}(f^{-1})^* \omega=i_{f_*(X^H)}(f^{-1})^* \omega = (f^{-1})^*i_{X^H}\omega=i_{X^{H \circ f^{-1}}}\omega.
  \end{equation*}
  Luego $(f^{-1})^* \omega = \omega$. Es decir, $f^{-1}$ (y por tanto $f$) es un simplectomorfismo.
\end{proof}

\begin{prop}[Teorema de Liouville]
  Los flujos hamiltonianos son familias uniparamétricas de simplectomorfismos, es decir, el flujo hamiltoniano preserva $\omega$. 
\end{prop}
\begin{proof}
  Si $X^H$ es un campo hamiltoniano, por la fórmula de Cartan,
  \begin{equation*}
    L_{X^H}\omega= \dd (i_{X^H}\omega) + i_{X^H}(\dd \omega) = \dd (\dd H) + 0 = 0.
  \end{equation*}
  Ahora, si $\varphi$ es el flujo de $X^H$,
  \begin{align*}
    \frac{d}{dt}(\varphi^*_t\omega_{\varphi_t(x)})&=\lim_{h\rightarrow 0}\frac{\varphi^*_{t+h}\omega_{\varphi_{t+h}(x)}-\varphi^*_t\omega_{\varphi_t(x)}}{h}\\ &=\varphi^*_t\left( \lim_{h\rightarrow 0}\frac{\varphi^*_h\omega_{\varphi_h(\varphi_t(x))}-\omega_{\varphi_t(x)}}{h}\right)\\ &=\varphi_t^*((L_X\omega)_{\varphi_t(x)})=\varphi^*_t(0)=0. 
  \end{align*}
  Por tanto, $\varphi^*_t\omega_{\varphi_t(x)}=\omega_x$.
\end{proof}
Una consecuencia inmediata del teorema de Liouville, es que los simplectomorfismos preservan el volumen de Liouville $\omega^n$.

A la vista de este resultado, vamos a probar un teorema muy importante sobre transformaciones que preservan el volumen:
\begin{prop}[Teorema de recurrencia de Poincaré \footnote{En inglés \textit{Poincaré recurrence theorem}. Nótese que aquí la palabra «recurrencia» no toma el significado habitual en matemáticas (por ejemplo en «construcción de sucesiones por recurrencia»), que se traduce del inglés \textit{recursion}. El diccionario Oxford define \textit{recursion} como \textit{``the repeated application of a recursive procedure or definition''}. Por otro lado, define \textit{recurrence} como \textit{``the fact of ocurring again''}, que podría traducirse también como «repetición» o «reaparición».}]
 Sea $D\subset \RR^n$ acotado y sea $g:D\rightarrow D$ que preserva el volumen. Entonces, para cualquier $U\subset D$ abierto, hay un punto $x\in D$ y un $n\in \NN$, $n>0$, tal que $g^n(x) \in U$.
\end{prop}
\begin{proof}
  Consideramos la familia
  \begin{equation*}
    \{g^nU | n \in \NN \}.
  \end{equation*}
  Todos estos conjuntos tienen el mismo volumen y, si no se cortaran en ningún punto, $D$ tendría volumen infinito. Por tanto, existen $k,l\in \NN$, $k>l$ tal que
  \begin{equation*}
    g^kU\cap g^lU \neq \varnothing,
  \end{equation*}
  luego $A=g^{k-l}U \cap U \neq \varnothing$. Entonces, dado $y\in A$, existen $x \in U$ y $n=k-l$ tal que $y=g^n(x)\in U$.
\end{proof}
\begin{ejemplo}
  \em
  Sea $D$ una circunferencia y $g$ la rotación de ángulo $\alpha$. Si $\alpha=2\pi(m/n)$, entonces $g^n$ es la identidad y el resultado es obvio. Pero, si $\alpha$ es un múltiplo irracional de $2\pi$, entonces, por el teorema de recurrencia de Poincaré, para todo $x\in D$ y para todo $\delta>0$ existe un $n\in \NN$ tal que $g^n(x) \in B_{\delta}(x)$.
  De aquí se sigue que, dado $x \in D$, el conjunto $\{g^k(x)|k\in \NN\}$ es denso en $D$. Más adelante veremos una aplicación de este ejemplo a mecánica hamiltoniana.

\end{ejemplo}

Para terminar la sección, haremos un breve comentario sobre cantidades conservadas.
\begin{defn}
  \em
  Sea $(M,H)$ un sistema hamiltoniano, una función $F:M \rightarrow \RR$ se dice que es una \emph{integral primera} del sistema o una \emph{constante del movimiento} si es constante a lo largo del flujo hamiltoniano. Esto es, si
  \begin{equation*}
    F(\varphi^H_t(x))=F(x)
  \end{equation*}
  para todo $t \geq 0$ y para todo $x \in M$.
  \begin{prop}[Ley de conservación de la energía]
    $H$ es una integral primera del sistema hamiltoniano $(M,H)$.
  \end{prop}
  \begin{proof}
    Basta hallar la derivada de $H$ en la dirección de $X^H$ y ver que es 0. En efecto, dado $x \in M$,
    \begin{equation*}
      \frac{d}{dt}\left( \varphi_t^H(x) \right)=d_xH(X_x^H) = \omega(X_x^H,X_x^H)=0,
    \end{equation*}
    ya que $d_xH=\omega(X_x^H,\bullet)$ y $\omega$ es antisimétrica.
  \end{proof}
\end{defn}

\section{Corchete de Poisson}\label{sec:poisson}
\begin{defn}[Corchete de Poisson]
  \em
  Sea $(M,\omega)$ una variedad simpléctica y $F,G: M \rightarrow \RR$. Se define el \emph{corchete de Poisson de $F$ y $G$} como la función
\begin{equation*}
  \pois{F}{G}(x) = \left.\frac{d}{dt}\right|_{t=0}G(\varphi^F_t(x)),
\end{equation*}
para cada $x \in M$.
\end{defn}
\begin{prop}\leavevmode
  \begin{enumerate}
    \item Una función $F:M \rightarrow \RR$ es una primera integral de $(M,H)$ si y sólo si $\pois{H}{F}$ es idénticamente nula.
    \item $\pois{F}{G} = \dd G (X^F)$.
    \item $\pois{F}{G} = \omega(X^ G, X^F)$.
    \item $\pois{ }{ }: \mathcal{C}^{\infty}(M) \times \mathcal{C}^{\infty}(M) \rightarrow \mathcal{C}^{\infty}(M)$ es una aplicación bilineal y antisimétrica. Es decir, 
      \begin{equation*}
	\pois{F}{G}  =  -\pois{G}{F} \\
      \end{equation*}
      y
      \begin{equation*}
	\pois{F}{\lambda_1 G_1 + \lambda_2 G_2}  =  \lambda_1 \pois{F}{G_1} + \lambda_2 \pois{F}{G_2},
      \end{equation*}
      para cualesquiera $\lambda_1, \lambda_2 \in \RR$.
    \item $\lie{X^F}{X^G} = X^{\pois{F}{G}}$.
  \end{enumerate}
\end{prop}
\begin{proof}\leavevmode
  \begin{enumerate}
    \item Inmediata de la definición de integral primera.
    \item Por la regla de la cadena y por el hecho de que $X^ F$ es el generador infinitesimal de $\varphi^F$.
    \item $\omega(X^G,X^F)=\dd G (X^F) = \pois{F}{G}$.
    \item Por (3) y porque $\omega$ es bilineal y antisimétrica.
    \item Por las fórmulas de Cartan:
      \begin{align*}
	i_{\lie{X^F}{X^G}} \omega & =  L_{X^F}(i_{X^G}\omega) - i_{X^G}(L_{X^F} \omega) \\ 
	& = \dd (i_{X^F}(i_{X^G}\omega)) + i_{X^F}(\dd (i_{X^G}\omega)) - i_{X^G}(\dd(i_{X^F}\omega) - i_{X^G}(i_{X^F}(\dd \omega)) \\
	& = \dd (\omega(X^G,X^F)) + i_{X^F}(\dd(\dd G)) - i_{X^G}(\dd (\dd F)) - 0 \\
	& = \dd (\pois{F}{G}) = i_{X^{\pois{F}{G}}}\omega.
      \end{align*}
  \end{enumerate}
\end{proof}

\begin{ejemplo}
  \em
  Consideremos el espacio simpléctico canónico $(\RR^{2n},\omega_n)$ y dos funciones $F,G:\RR^{2n} \rightarrow \RR$. Obtenemos entonces la definición clásica del corchete de Poisson:
  \begin{align*}
    \pois{F}{G} & = \dd G (X^F) = \left( \sum_{i=1}^n \parcial{G}{q_i} \dd q_i + \parcial{G}{p_i} \dd p_i \right) \left( \sum_{i=1}^n \parcial{F}{p_i} \deriv{q_i} - \parcial{F}{q_i} \deriv{p_i}\right) \\ 
    & =\sum_{i=1}^n \parcial{F}{p_i} \parcial{G}{q_i} - \parcial{F}{q_i} \parcial{G}{p_i}.
  \end{align*}
  Observamos también las relaciones
  \begin{align*}
    \pois{F}{q_i} &= \parcial{F}{p_i} \\
    \pois{F}{p_i} &= -\parcial{F}{q_i},
  \end{align*}
  lo que deja las ecuaciones de Hamilton en una forma más simple:
  \begin{equation*}
    \left\lbrace
    \begin{array}{l}
      \dot{q_i}=\pois{H}{q_i} \\
      \dot{p_i}=\pois{H}{p_i}.
    \end{array}
    \right.
  \end{equation*}
  De aquí también obtenemos las \emph{relaciones de conmutación canónicas}:
  \begin{equation*}
    \pois{p_i}{q_j}=\delta_{ij}.
  \end{equation*}
\end{ejemplo}

  El corchete de Poisson permite probar la versión hamiltoniana de uno de los resultados más importantes de toda la Física, debido a Emmy Noether, que relaciona las simetrías de un sistema con sus cantidades conservadas.
  \begin{prop}[Teorema de Noether]
    Sea $(M,H)$ un sistema hamiltoniano y $F:M \rightarrow \RR$. Si $H$ es constante a lo largo de $X^F$, entonces $F$ es una integral primera de $(M,H)$. 
  \end{prop}
  \begin{proof}
    Como $H$ es constante a lo largo de $X^F$, es una integral primera de $(M,F)$, luego $\pois{F}{H}=0$. Ahora, $\pois{H}{F}=-\pois{F}{H}=0$, luego $F$ es una integral primera de $(M,H)$. 
  \end{proof}
Otra forma de ver este mismo teorema es la siguiente:
\begin{prop}[Otra forma del teorema de Noether]
  Sea $(M,\omega)$ una variedad simpléctica y sean $X^F$ y $X^G$ campos hamiltonianos en $M$. Los dos campos conmutan (y, por tanto, lo hacen los flujos que generan) si y sólo si $\pois{F}{G}$ es constante. 
\end{prop}
\begin{proof}
  Sea $\pois{F}{G}=a \in \RR$, entonces
  \begin{equation*}
    \lie{X^F}{X^G}= X^{\pois{F}{G}} = X^a=0.
  \end{equation*}
\end{proof}

Podemos ver también que el corchete de Poisson es un corchete de Lie sobre $\mathcal{C}^{\infty}(M)$, ya que cumple la identidad de Jacobi:
\begin{prop}[Identidad de Jacobi]
 Sea $(M,\omega)$ una variedad simpléctica y sean  $A,B,C:M \rightarrow \RR$, entonces
 \begin{equation*}
   \pois{\pois{A}{B}}{C}+ \pois{\pois{B}{C}}{A} + \pois{\pois{C}{A}}{B}=0
 \end{equation*}
 \begin{proof}
   Tenemos 
   \begin{align*}
     \lie{X^A}{X^B} (C) & = X^A\circ X^B (C) - X^B\circ X^A (C)  = X^A(\pois{B}{C})-X^B(\pois{A}{C}) \\
     &= \pois{A}{\pois{B}{C}}-\pois{B}{\pois{A}{C}}.
   \end{align*}
   Mientras que
   \begin{equation*}
     \lie{X^A}{X^B}(C)=X^{\pois{A}{B}}(C)=\pois{\pois{A}{B}}{C}.
   \end{equation*}
   Tenemos entonces
   \begin{equation*}
     \pois{\pois{A}{B}}{C}=\pois{A}{\pois{B}{C}}-\pois{B}{\pois{A}{C}}=-\pois{\pois{B}{C}}{A}-\pois{\pois{C}{A}}{B}.
   \end{equation*}
   Pasando el segundo término al otro lado obtenemos la identidad de Jacobi.
 \end{proof}
\end{prop}
\begin{corol}[Teorema de Poisson]
  Si $F_1, F_2$ son integrales primeras de $(M,H)$, entonces $\pois{F_1}{F_2}$ es también una integral primera de $(M,H)$.
\end{corol}
\begin{proof}
  Por la identidad de Jacobi,
  \begin{equation*}
    \pois{ \pois{F_1}{F_2}}{ H}= \pois{F_1}{ \pois{F_2}{H}} + \pois{F_2}{ \pois{H}{F_1} }=0,
  \end{equation*}
  ya que $F_1,F_2$ son integrales primeras.
\end{proof}

De la bilinealidad, la antisimetría y la identidad de Jacobi obtenemos inmediatamente el siguiente resultado:
\begin{corol}
  $(\mathcal{C}^{\infty}(M),\pois{ }{ })$ es un álgebra de Lie.  
\end{corol}

Por último, vamos a dar otra caracterización clásica de los simplectomorfismos, que preservan el corchete de Poisson.
\begin{prop}
  Sea $M$ una variedad simpléctica y $f:M\rightarrow M$. La función $f$ es un simplectomorfismo si y sólo si para cualesquiera $F,G: M \rightarrow \RR$, $\pois{F}{G} \circ f^{-1}=\pois{F\circ f^{-1}}{G \circ f^{-1}}$.
\end{prop}
\begin{proof}\leavevmode
  Tenemos,
  \begin{equation*}
    \pois{F}{G} \circ f^{-1} = X^{F}(G) \circ f^{-1} = f_* X^F (G \circ f^{-1})
  \end{equation*}
  \begin{equation*}
    \pois{F\circ f^{-1}}{G \circ f^{-1}} = X^{F\circ f^{-1}}(G \circ f^{-1}).
  \end{equation*}

  Por tanto, $\pois{F}{G} \circ f^{-1}=\pois{F\circ f^{-1}}{G \circ f^{-1}}$ si y sólo si $X^{F \circ f^{-1}}= f_* X^{F}$, para cada $F: M \rightarrow \RR$.
\end{proof}

